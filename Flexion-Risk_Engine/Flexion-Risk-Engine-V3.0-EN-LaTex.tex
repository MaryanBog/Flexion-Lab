\documentclass[12pt]{article}

\usepackage[a4paper,margin=1in]{geometry}
\usepackage{amsmath, amssymb}
\usepackage{hyperref}
\usepackage{titlesec}

\title{Flexionization Risk Engine (FRE) V3.0 \\ 
\large Short Formal Specification}
\author{Maryan Bogdanov}
\date{2025}

\begin{document}

\maketitle
\tableofcontents
\newpage

% ========================================================
\section*{Abstract}

The Flexionization Risk Engine (FRE) V3.0 is a multidimensional structural risk model 
based on the Flexion Framework. FRE analyzes stability, deviation, contraction, tension, 
and collapse-prevention through the state vector 
\[
X = (\Delta, \Phi, M, \kappa),
\]
and the five-dimensional deviation geometry
\[
\vec{\Delta} = (\Delta_m, \Delta_L, \Delta_H, \Delta_R, \Delta_C) \in \mathbb{R}^5.
\]
Version 3.0 introduces a diagonal corrective operator, equilibrium pressure (FXI), 
capacity boundaries, structural zones, and a unified collapse boundary (SB).  
This short formal specification summarizes the core mathematics and architectural 
principles of FRE V3.0 for archival publication.

\section{Introduction}

FRE V3.0 generalizes structural risk analysis to a fully geometric, 
multidimensional framework.  
Traditional engines rely on probability, volatility, or external triggers, 
while FRE derives all risk from internal structural deformation.  

Risk in FRE arises from:
\begin{itemize}
    \item deviation magnitude ($\Delta$),
    \item structural tension ($\Phi$),
    \item irreversible memory ($M$),
    \item weakening contractivity ($\kappa$).
\end{itemize}

FRE V3.0 provides domain-agnostic guarantees of stability, boundedness, monotonicity, 
and collapse-prevention. The system is fully deterministic and operates strictly inside 
the Viability Domain.

\section{State Representation}

The system state is defined as:
\[
X = (m, L, H, R, C),
\]
with the five-dimensional deviation vector:
\[
\vec{\Delta} = D(X) = 
\big(D_m(X),\, D_L(X),\, D_H(X),\, D_R(X),\, D_C(X)\big).
\]

All deviation components satisfy:
\begin{itemize}
    \item continuity,
    \item monotonicity,
    \item boundedness,
    \item admissibility for all system states.
\end{itemize}

\section{Equilibrium Pressure (FXI)}

The equilibrium indicator is defined as:
\[
FXI = F(\vec{\Delta}),
\]
a scalar field describing how deviation contributes to structural tension.

Near equilibrium: $FXI \to 0$.  
Near boundary conditions: $FXI$ becomes large, indicating rising collapse pressure.

FXI drives contraction strength and determines the behavior of corrective operators.

\section{Diagonal Correction Operator}

FRE V3.0 introduces a diagonal structure:
\[
\vec{C} =
\big(
E_m(\Delta_m),\, 
E_L(\Delta_L),\,
E_H(\Delta_H),\,
E_R(\Delta_R),\,
E_C(\Delta_C)
\big).
\]

Each component operator $E_i(\cdot)$ satisfies:
\begin{align*}
&\text{continuity}, \\
&\text{monotonicity: }
\Delta>0 \Rightarrow E_i(\Delta)<0,\quad
\Delta<0 \Rightarrow E_i(\Delta)>0, \\
&\text{contraction: }
|E_i(\Delta)| < k_i |\Delta|,\ 0 < k_i < 1, \\
&\text{boundedness: }
|E_i(\Delta)| \le L_i.
\end{align*}

The operator is stable, interpretable, and compatible with global FXI.

\section{System Evolution Rule}

The structural update rule:

\[
X_{t+1} = X_t + E(\vec{\Delta}_t),
\]

with
\[
\vec{\Delta}_t = D(X_t).
\]

The evolution is:
\begin{itemize}
    \item continuous,
    \item bounded,
    \item contracting,
    \item globally admissible,
    \item strictly monotonic toward equilibrium.
\end{itemize}

\section{Structural Zones}

The deviation space is partitioned into five zones:

\begin{enumerate}
    \item \textbf{CSZ — Core Stability Zone}: strong contraction, minimal deviation.
    \item \textbf{SAZ — Safe Adjustment Zone}: fully correctable dynamics.
    \item \textbf{PRZ — Peripheral Risk Zone}: sensitivity increases.
    \item \textbf{CZ — Critical Zone}: contraction fragile, FXI curvature rises.
    \item \textbf{SB — Structural Boundary}: 
    \[
    \|\vec{\Delta}\|_W = C_{\text{global}},
    \]
    beyond which the system collapses.
\end{enumerate}

SB marks the geometric limit of admissible structural behavior.

\section{Collapse Boundary (SB)}

The Structural Boundary (SB) is defined by:
\[
\|\vec{\Delta}\|_W \to C_{\text{global}}.
\]

Crossing SB results in:
\begin{itemize}
    \item irreversible breakdown dynamics,
    \item loss of contractivity,
    \item exit from the Viability Domain.
\end{itemize}

SB is the formal collapse boundary in FRE V3.0.

\section{Key Results}

FRE V3.0 guarantees:

\begin{itemize}
    \item \textbf{Contraction}: every admissible trajectory moves toward equilibrium.
    \item \textbf{Stability}: deviations remain bounded inside the structural domain.
    \item \textbf{Continuity}: no jumps, oscillations, or undefined states.
    \item \textbf{Recoverability}: geometric signals indicate stabilization after stress.
    \item \textbf{Collapse Prediction}: FXI curvature reveals pre-collapse patterns.
\end{itemize}

These results form the universal theoretical core of the FRE architecture.

\section{Conclusion}

This short formal specification summarizes the mathematical foundations and structural 
architecture of FRE V3.0.  
The engine defines deviation geometry, contraction operators, stability zones, capacity 
boundaries, and collapse dynamics.  
FRE V3.0 is the complete theoretical baseline for all future structural risk systems in 
Flexion Science.

\end{document}
