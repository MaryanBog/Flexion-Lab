\documentclass[12pt]{article}

% Packages
\usepackage{amsmath, amssymb}
\usepackage{graphicx}
\usepackage{hyperref}
\usepackage{geometry}
\usepackage{titlesec}
\usepackage{enumitem}

\geometry{margin=1in}

\title{
    \textbf{Flexion Trading Theory:}\\
    \textbf{A Structural Framework for Geometric Market Dynamics}
}
\author{Marian Bohdanov}
\date{\today}

\begin{document}

\maketitle

\begin{abstract}
    Flexion Trading Theory (FTT) introduces a purely structural and deterministic
    framework for understanding market dynamics by interpreting the price series
    $P(t)$ as a geometric object with measurable curvature and torsion.  
    Instead of modelling volatility, patterns, or statistical distributions,
    FTT derives all trading decisions from discrete flexion derivatives---the
    first-order price change $\Delta P(t)$, the curvature $C(t)$, and the torsion
    $\Delta S(t)$---computed on an adaptive structural window.
    
    The theory defines two fundamental geometric events: the \textit{Flexion Entry
    Window} (FEW), marking the initiation of a coherent bending regime, and the
    \textit{Flexion Break} (FXB), marking torsional collapse and mandatory exit.
    Both events arise solely from the intrinsic structure of $P(t)$ and require no
    external indicators, thresholds, probabilistic filters, or empirical tuning.
    
    By combining flexion derivatives, structural orientation, and adaptive
    stability filters, FTT provides a complete decision function (FTT\_API) that
    maps observable ticker data to deterministic trading actions.  
    This theoretical foundation enables the Flexion Trading Runtime Engine (FTRE),
    a platform-independent executable specification that implements FTT without
    modification.
    
    FTT establishes a universal geometric description of market impulses, offering
    a reproducible, deterministic, and structurally grounded alternative to
    classical technical analysis.  
\end{abstract}    

\section{Introduction}

Financial markets are commonly modelled through statistical, stochastic, or
indicator-based frameworks that rely on assumptions about volatility,
distributional properties, or price patterns.  
Flexion Trading Theory (FTT) departs fundamentally from these approaches by
treating the market not as a stochastic generator of prices but as a
\emph{geometric structure} evolving through time.

In FTT, the only observable quantity is the discrete price sequence $P(t)$.
From this sequence, the theory derives a hierarchy of \emph{flexion
derivatives}---the first derivative $\Delta P(t)$ (directional motion), the
second derivative $C(t)$ (curvature), and the third derivative $\Delta S(t)$
(torsion).  
These derivatives enable a precise structural interpretation of price
evolution that is independent of volatility modeling, pattern recognition,
machine learning, or traditional technical indicators.

The core insight of FTT is that market impulses can be described entirely by
geometric transitions in curvature and torsion.  
Two events define the complete structural lifecycle of an impulse:  
(1) the \emph{Flexion Entry Window} (FEW), marking the birth of a coherent
bending regime aligned with structural orientation, and  
(2) the \emph{Flexion Break} (FXB), marking the torsional collapse of that
regime.  

This perspective yields a deterministic trading framework in which signals
arise solely from the intrinsic geometry of $P(t)$, without recourse to
external parameters, empirical adjustment, or probabilistic filtering.  
FTT thus stands as a fully reproducible, purely structural model of trading
decision-making.

Finally, the theory provides a formal mapping to the Flexion Trading Runtime
Engine (FTRE), which implements FTT as a platform-independent computational
engine.  
This ensures that the structural logic of the theory remains faithful and
consistent across trading environments, making FTT both a scientific framework
and a practical foundation for automated trading systems.

\section{Observable Domain}

Flexion Trading Theory (FTT) operates exclusively on observable market data.
No latent variables, statistical estimates, or external indicators enter the
theory.  
At every discrete event index $t$, the sole observable is the last traded
price $P(t)$, from which all structural quantities are derived.

\subsection{Price Series}

The price series is defined as a discrete sequence
\[
P(t), \qquad t \in \mathbb{Z}^{+},
\]
where each index corresponds to a market tick.  
FTT does not assume continuity, differentiability, or stochastic properties of
the price series.  
Only the ordering and numerical values of $P(t)$ are relevant for the theory.

\subsection{Event Time Basis}

The temporal structure of FTT is based on event order rather than physical
time.  
The sequence $t = 0,1,2,\ldots$ denotes successive observed price updates,
regardless of their wall-clock spacing.  
Thus, FTT remains invariant under irregular tick arrival, latency variations,
and session-specific timing effects.

\subsection{Internal Structural Window}

FTT uses a finite rolling window of past observations,
\[
W(t) = \{ P(t-k) \mid k = 0,1,\dots, W(t) \},
\]
to compute flexion derivatives and structural quantities.  
Unlike fixed-size windows common in technical analysis, $W(t)$ is \emph{adaptive}
and determined by structural stability conditions (Section~\ref{sec:filters}).

This windowing mechanism ensures that structural calculations remain meaningful
even under noise or fluctuating market regimes.

\subsection{Exclusion of External Inputs}

FTT explicitly forbids the use of:

\begin{itemize}[noitemsep]
    \item volume data,
    \item OHLC aggregations,
    \item volatility estimates,
    \item statistical indicators,
    \item moving averages,
    \item configurable thresholds or empirical parameters.
\end{itemize}

All structural features arise exclusively from observable price data and its
discrete variations.  
This restriction ensures full theoretical purity and reproducibility.

\subsection{Closed Observable System}

The observable domain of FTT is a closed system:
\[
P(t) \;\longrightarrow\; \Delta P(t),\, C(t),\, \Delta S(t)
\;\longrightarrow\; K(t)
\;\longrightarrow\; \text{Events}
\;\longrightarrow\; \text{FTT\_API}.
\]

Every quantity used by the theory is either directly observable or derived
from $P(t)$.  
No external assumptions or market-specific features influence the structural
logic of the model.

\section{Flexion Derivatives}
\label{sec:flexion-derivatives}

Flexion Trading Theory interprets market structure through discrete geometric
derivatives of the observable price series.  
These derivatives capture directional motion, curvature, and torsional
instability, forming the minimal and complete structural basis of FTT.

All derivatives are defined on the adaptive structural window $W(t)$ described
in Section~\ref{sec:filters}.  
No smoothing, averaging, interpolation, or external filtering is permitted.

\subsection{First Flexion Derivative: $\Delta P(t)$}

The first derivative measures instantaneous directional change:
\[
\Delta P(t) = P(t) - P(t-1).
\]

Properties:
\begin{itemize}[noitemsep]
    \item $\mathrm{sign}(\Delta P(t))$ indicates local upward or downward motion,
    \item $|\Delta P(t)|$ reflects impulse intensity,
    \item provides the foundation for higher-order derivatives.
\end{itemize}

$\Delta P(t)$ is the elemental geometric primitive from which curvature and
torsion emerge.

\subsection{Second Flexion Derivative: $C(t)$}

The second derivative,
\[
C(t) = \Delta P(t) - \Delta P(t-1),
\]
represents geometric curvature of the price trajectory.

Interpretation:
\begin{itemize}[noitemsep]
    \item $C(t) > 0$ --- upward (convex) bending,
    \item $C(t) < 0$ --- downward (concave) bending,
    \item $C(t) = 0$ --- locally linear structural motion.
\end{itemize}

Curvature is the primary driver of structural orientation $K(t)$.

\subsection{Third Flexion Derivative: $\Delta S(t)$}

The third derivative,  
\[
\Delta S(t) = C(t) - C(t-1),
\]
measures torsion: the rate at which curvature changes direction.

Interpretation:
\begin{itemize}[noitemsep]
    \item peaks of $\Delta S(t)$ signify structural instability,
    \item extremal values correspond to torsion collapse (FXB),
    \item zero-crossings mark transitions between bending regimes.
\end{itemize}

$\Delta S(t)$ is the core geometric quantity governing exit logic.

\subsection{Discrete Derivative Properties}

All flexion derivatives satisfy strict locality:

\[
\Delta P(t),\, C(t),\, \Delta S(t)
    \quad \text{depend only on} \quad
    P(t), P(t-1), P(t-2), P(t-3).
\]

No additional memory or long-range dependencies are allowed.  
This guarantees:
\begin{itemize}[noitemsep]
    \item determinism,
    \item reproducibility across platforms,
    \item independence from tick-rate variation.
\end{itemize}

\subsection{Minimal Structural Basis}

The triplet
\[
(\Delta P(t),\, C(t),\, \Delta S(t))
\]
forms the minimal complete set of structural descriptors of price evolution.

No additional derivatives or transformations are required to define:
\begin{itemize}[noitemsep]
    \item structural orientation,
    \item initiation events (FEW),
    \item collapse events (FXB),
    \item the full decision function (FTT\_API).
\end{itemize}

Thus the flexion derivatives constitute the canonical geometric representation
of market structure in FTT.

\section{Structural Orientation}
\label{sec:orientation}

Structural orientation $K(t)$ defines the dominant geometric direction of
market motion.  
It is derived exclusively from curvature $C(t)$ and reflects the persistent
bias of bending within the adaptive structural window $W(t)$.

Orientation acts as a directional stability mechanism:  
it ensures that only structurally meaningful bending regimes produce valid
entry events.  
Noise-driven curvature fluctuations, lacking geometric persistence, are
suppressed by $K(t)$.

\subsection{Definition of Structural Orientation}

FTT defines orientation as the sign of cumulative curvature across the adaptive
window:
\[
K(t) = \mathrm{sign}\!\left( \sum_{i=0}^{W(t)} C(t-i) \right).
\]

Thus:
\[
K(t) =
\begin{cases}
+1, & \text{if upward curvature dominates}, \\
-1, & \text{if downward curvature dominates}, \\
0,  & \text{if curvature lacks a stable direction}.
\end{cases}
\]

This definition ensures that $K(t)$ emerges directly from the underlying
geometry of $P(t)$.

\subsection{Geometric Interpretation}

\paragraph{Upward Orientation ($K=+1$).}
Curvature values tend to be positive, indicating a convex structural regime.
Upward flexion signals (FEW\_UP) are permitted.

\paragraph{Downward Orientation ($K=-1$).}
Curvature values tend to be negative, indicating a concave structural regime.
Downward flexion signals (FEW\_DN) are permitted.

\paragraph{Neutral Orientation ($K=0$).}
Curvature lacks a stable sign; the structure is transitional or dominated by
micro-noise.  
All FEW events are suppressed in this regime.

\subsection{Role of Orientation in Event Logic}

Orientation imposes necessary constraints on entry logic:

\begin{itemize}[noitemsep]
    \item FEW may occur only when $K(t) \neq 0$,
    \item FEW direction must match $K(t)$,
    \item opposite-direction curvature changes are ignored as noise.
\end{itemize}

This geometric filter prevents premature or invalid entries triggered by
microscopic curvature fluctuations.

\subsection{Relation to Torsion and Collapse}

While orientation governs \emph{entry} (FEW) conditions, it has no influence
on \emph{exit} (FXB) conditions.  
Torsion collapse is a symmetric geometric event detected solely by $\Delta S(t)$.

The division of roles:
\begin{itemize}[noitemsep]
    \item $K(t)$ --- determines directional structural validity,
    \item $\Delta S(t)$ --- determines collapse of bending,
\end{itemize}
ensures that FTT maintains both geometric purity and practical structural
stability.

\subsection{Deterministic and Parameter-Free}

Orientation is computed without:
\begin{itemize}[noitemsep]
    \item smoothing averages,
    \item thresholds,
    \item tunable parameters,
    \item probabilistic estimates.
\end{itemize}

It depends only on the discrete curvature values within $W(t)$ and is therefore
deterministic, reproducible, and invariant across platforms.

\section{Structural Events}
\label{sec:events}

Flexion Trading Theory defines market transitions through two fundamental
structural events derived exclusively from the geometric behavior of the
flexion derivatives.  
These events describe the birth and collapse of coherent bending regimes in
the price trajectory:

\begin{itemize}[noitemsep]
    \item \textbf{FEW} --- Flexion Entry Window (structural initiation),
    \item \textbf{FXB} --- Flexion Break (structural collapse).
\end{itemize}

No additional event types exist in the theory.  
FEW marks the beginning of a structurally coherent impulse, whereas FXB marks
its termination at torsion extremum.  
Together they form the complete structural lifecycle of a trade.

\subsection{FEW: Flexion Entry Window}

A FEW event represents the moment when the market enters a new bending regime
aligned with the dominant structural orientation $K(t)$.

\subsubsection*{Formal Definition}

A FEW event occurs at time $t$ if and only if all the following conditions hold:

\begin{enumerate}[label=(\arabic*), noitemsep]
    \item \textbf{Orientation Alignment:}
    \[
    \mathrm{sign}(C(t)) = K(t), \qquad K(t) \neq 0.
    \]

    \item \textbf{Curvature Emergence:}
    \[
    C(t-1) = 0, \qquad C(t) \neq 0.
    \]
    This ensures FEW corresponds to the \emph{first} moment of bending.

    \item \textbf{Curvature Strengthening:}
    \[
    C(t) > C(t-1) \quad \text{if } K(t)=+1,
    \]
    \[
    C(t) < C(t-1) \quad \text{if } K(t)=-1.
    \]
\end{enumerate}

Thus:
\[
\text{FEW\_UP} \longleftrightarrow K(t)=+1,
\qquad
\text{FEW\_DN} \longleftrightarrow K(t)=-1.
\]

\subsubsection*{Interpretation}

FEW corresponds to:
\begin{itemize}[noitemsep]
    \item the birth of curvature,
    \item the initiation of geometric bending,
    \item the structurally valid entry into a market impulse.
\end{itemize}

FEW is never triggered by:
\begin{itemize}[noitemsep]
    \item price patterns,
    \item volatility changes,
    \item statistical indicators,
    \item oscillatory micro-noise.
\end{itemize}

It arises purely from the intrinsic geometry of $P(t)$.

\subsection{FXB: Flexion Break}

FXB marks the moment when the bending regime becomes unstable and collapses.
It is defined by the behavior of torsion $\Delta S(t)$.

\subsubsection*{Formal Definition}

A FXB event occurs at time $t$ if and only if:

\begin{enumerate}[label=(\arabic*), noitemsep]
    \item \textbf{Torsion Extremum:} $\Delta S(t)$ is a strict local extremum:
    \[
    \Delta S(t-1) < \Delta S(t) > \Delta S(t+1)
    \quad \text{(peak)},
    \]
    \[
    \Delta S(t-1) > \Delta S(t) < \Delta S(t+1)
    \quad \text{(trough)}.
    \]

    \item \textbf{Non-Degeneracy:}
    \[
    \Delta S(t) \neq \Delta S(t-1), \qquad
    \Delta S(t) \neq \Delta S(t+1).
    \]
\end{enumerate}

FXB is \emph{independent} of orientation:
\[
K(t) \;\text{plays no role in FXB detection}.
\]

\subsubsection*{Interpretation}

FXB corresponds to:
\begin{itemize}[noitemsep]
    \item the point of maximum torsional instability,
    \item the geometric collapse of bending,
    \item the mandatory structural exit of a trade.
\end{itemize}

FXB always terminates the current structural cycle.

\subsection{Structural Cycle Consistency}

The two events form a strict geometric sequence:
\[
\text{FEW}(t_0) \;\longrightarrow\; \text{FXB}(t_1),
\qquad t_1 > t_0.
\]

The following invariants must hold:
\begin{itemize}[noitemsep]
    \item no FEW may occur after FEW and before FXB,
    \item no FXB may occur without a preceding FEW,
    \item each FEW must eventually be followed by an FXB,
    \item exactly one FEW and one FXB per structural cycle.
\end{itemize}

These invariants guarantee a well-defined geometric lifecycle of market
impulses.

\section{Flexion Windows and Stability Filters}
\label{sec:filters}

Flexion Trading Theory employs an adaptive structural window and a set of
deterministic stability filters to ensure that structural events (FEW, FXB)
are computed on a geometrically meaningful basis rather than on raw
tick-level noise.  
These mechanisms do not introduce smoothing, heuristics, or empirical tuning;
instead, they preserve the structural purity of the theory while enforcing
numerical stability.

The adaptive window $W(t)$ governs the locality of structural calculations,
while the filters $\mu(t)$, $M(t)$, and $\Theta(t)$ ensure that FEW events are
triggered only when market geometry exhibits coherent bending.

\subsection{Adaptive Structural Window $W(t)$}

The structural window $W(t)$ defines how many past observations are used to
evaluate curvature, orientation, and stability.  
Unlike fixed lookback windows typical of technical indicators, $W(t)$ expands
or contracts based on structural consistency.

\subsubsection*{Definition}

$W(t)$ is defined as the minimum window length for which curvature sign
stabilizes:
\[
W(t)
= \min \left\{ k \;:\;
\mathrm{sign}\!\left( \sum_{i=0}^{k} C(t-i) \right)
=
\mathrm{sign}\!\left( \sum_{i=0}^{k+1} C(t-i) \right)
\right\}.
\]

Interpretation:
\begin{itemize}[noitemsep]
    \item small $W(t)$ (3--7 ticks) indicates stable bending,
    \item large $W(t)$ indicates geometric inconsistency or noise.
\end{itemize}

The window adapts purely from geometric structure and introduces no external
parameters.

\subsection{Magnitude Filter $\mu(t)$}

The magnitude filter ensures that only structurally meaningful impulses
trigger FEW events.  
It is defined as:
\[
\mu(t) = |\Delta P(t)|.
\]

The FEW event is allowed only when:
\[
\mu(t) > \mu_{\min}(W(t)),
\]
where $\mu_{\min}(W)$ grows with window size.  
This requirement suppresses micro-fluctuations that occur during high-noise
conditions.

\subsection{Curvature Stability Filter $M(t)$}

The curvature stability filter measures the average magnitude of curvature
over the structural window:
\[
M(t) = \frac{1}{W(t)+1}
\sum_{i=0}^{W(t)} |C(t-i)|.
\]

FEW is permitted only when instantaneous curvature exceeds a scaled stability
threshold:
\[
|C(t)| > \alpha \, M(t),
\]
with $\alpha$ representing the minimal structural dominance required for
coherent bending.  
This filter prevents FEW from triggering during geometric inconsistencies.

\subsection{Temporal Coherence Filter $\Theta(t)$}

The temporal coherence filter ensures that torsion dynamics remain meaningful
within the structural window.  
It is defined as:
\[
\Theta(t) = \frac{W(t)}{|\Delta S(t)| + \varepsilon},
\]
where $\varepsilon$ is the smallest nonzero torsion increment imposed by the
discrete price grid.

Interpretation:
\begin{itemize}[noitemsep]
    \item small $\Theta(t)$ indicates strong and coherent torsion,
    \item large $\Theta(t)$ indicates degeneracy or noise.
\end{itemize}

FEW is suppressed when $\Theta(t)$ exceeds an upper coherence bound
$\Theta_{\max}$.

\subsection{Summary of Stability Filters}

A FEW event is valid only when:
\begin{itemize}[noitemsep]
    \item orientation alignment is satisfied ($K(t) \neq 0$),
    \item curvature direction agrees with orientation,
    \item curvature is strengthening,
    \item impulse magnitude is sufficient: $\mu(t) > \mu_{\min}(W)$,
    \item curvature is coherent: $|C(t)| > \alpha M(t)$,
    \item torsion is temporally consistent: $\Theta(t) < \Theta_{\max}$.
\end{itemize}

These filters do not alter the structural meaning of the derivatives; rather,
they guarantee that flexion events reflect real geometric transitions rather
than microstructural noise.

\section{Flexion Trading Structural Cycle}
\label{sec:cycle}

The Flexion Trading Structural Cycle (FTSC) is the complete geometric lifecycle
of a market impulse as defined by Flexion Trading Theory.  
It encapsulates the transition from structural initiation to structural
collapse through the interaction of curvature, torsion, and orientation.

The cycle consists of three deterministic stages:

\begin{enumerate}[noitemsep]
    \item \textbf{Structural Initiation (FEW)},  
    \item \textbf{Structural Propagation},  
    \item \textbf{Structural Collapse (FXB)}.
\end{enumerate}

This cycle governs the lifecycle of every valid trade in FTT.

\subsection{Stage 1: Structural Initiation (FEW)}

Structural initiation begins when the market enters a coherent bending regime
aligned with the dominant orientation $K(t)$.

A FEW event requires:
\begin{itemize}[noitemsep]
    \item curvature emerging from linearity ($C(t-1)=0$, $C(t)\neq0$),
    \item curvature strengthening in the direction of $K(t)$,
    \item all stability filters satisfied,
    \item nonzero orientation ($K(t)\neq0$).
\end{itemize}

At FEW:
\begin{itemize}[noitemsep]
    \item the bending regime begins,
    \item the structural impulse becomes coherent,
    \item a trade is \emph{opened} in the direction of $K(t)$.
\end{itemize}

Thus:
\[
\text{FEW\_UP} \rightarrow \text{long position}, \qquad
\text{FEW\_DN} \rightarrow \text{short position}.
\]

\subsection{Stage 2: Structural Propagation}

During propagation, the market structure evolves in the direction established
at FEW.  
The structure remains coherent under the following invariants:

\begin{enumerate}[label=(\arabic*), noitemsep]
    \item orientation must retain sign: $\mathrm{sign}(K(t)) = \mathrm{sign}(K(t_0))$,
    \item curvature must preserve direction relative to $K(t)$,
    \item no torsion extremum may occur ($\Delta S(t)$ not at local peak/trough),
    \item stability filters must remain satisfied.
\end{enumerate}

Propagation represents the geometric ``life'' of the structural impulse.  
No additional entries may occur during this phase.

\subsection{Stage 3: Structural Collapse (FXB)}

Structural collapse occurs when torsion reaches a strict local extremum,
indicating instability in the curvature regime.

A FXB event is detected when:
\[
\Delta S(t-1) < \Delta S(t) > \Delta S(t+1)
\quad \text{or} \quad
\Delta S(t-1) > \Delta S(t) < \Delta S(t+1),
\]
with non-degeneracy constraints
\[
\Delta S(t) \neq \Delta S(t-1), \qquad
\Delta S(t) \neq \Delta S(t+1).
\]

At FXB:
\begin{itemize}[noitemsep]
    \item curvature coherence collapses,
    \item the structural impulse terminates,
    \item the open position must be closed.
\end{itemize}

FXB is direction-agnostic and independent of $K(t)$.

\subsection{Cycle Invariants}

Each structural cycle satisfies:

\begin{itemize}[noitemsep]
    \item exactly one FEW per cycle,
    \item exactly one FXB per cycle,
    \item FEW must precede FXB,
    \item no FEW may occur between FEW and FXB,
    \item no FXB may occur before an active FEW.
\end{itemize}

These invariants ensure that the FTSC forms a closed and deterministic
geometric system:
\[
\text{FEW} \;\longrightarrow\; \text{Propagation} \;\longrightarrow\; \text{FXB}.
\]

\section{Formal Decision Function (FTT\_API)}
\label{sec:api}

The Formal Decision Function, denoted FTT\_API, is the canonical mapping from
observable market data to discrete structural trading signals.  
It provides a complete, deterministic rule set for interpreting geometric
transitions in the price series using only flexion derivatives and the
adaptive structural filters.

FTT\_API outputs exactly one action at each event index $t$:
\[
\text{FTT\_API}(t) \in \{0, 1, 2, 3\},
\]
representing:
\[
0=\text{NONE},\qquad 
1=\text{FEW\_UP},\qquad
2=\text{FEW\_DN},\qquad
3=\text{FXB}.
\]

\subsection{Input Domain}

The only external input to FTT\_API is the observable price sequence
\[
P(t),\; P(t-1),\ldots, P(t-W(t)).
\]

All internal structural quantities used by the API are computed from $P(t)$:
\[
\Delta P(t),\; C(t),\; \Delta S(t),\; K(t),\; W(t),\; 
\mu(t),\; M(t),\; \Theta(t).
\]

These quantities are never externally supplied or tuned.

\subsection{Event Evaluation Order}

FTT imposes a strict priority hierarchy when evaluating structural events:

\begin{enumerate}[label=(\arabic*), noitemsep]
    \item \textbf{FXB} --- collapse event (highest priority),
    \item \textbf{FEW} --- initiation event,
    \item \textbf{NONE} --- fallback when no structural transition occurs.
\end{enumerate}

The priority ordering ensures that torsion collapse (FXB) always overrides
initiation (FEW), reflecting its structural significance.

\subsection{FXB Logic}

A FXB event is triggered when torsion reaches a strict extremum:
\[
\Delta S(t-1) < \Delta S(t) > \Delta S(t+1)
\]
or
\[
\Delta S(t-1) > \Delta S(t) < \Delta S(t+1),
\]
with non-degeneracy constraints:
\[
\Delta S(t)\neq\Delta S(t-1), \qquad
\Delta S(t)\neq\Delta S(t+1).
\]

If FXB is detected:
\[
\text{FTT\_API}(t) = 3.
\]

\subsection{FEW Logic}

A FEW event is triggered when curvature begins a coherent bending regime in
the direction of $K(t)$ and all stability filters are satisfied.

Formally:

\begin{enumerate}[label=(\arabic*), noitemsep]
    \item $\mathrm{sign}(C(t)) = K(t)$ and $K(t)\neq0$,
    \item $C(t-1)=0$, $C(t)\neq0$,
    \item strengthening condition:
    \[
    C(t) > C(t-1) \quad \text{if } K(t)=+1,
    \]
    \[
    C(t) < C(t-1) \quad \text{if } K(t)=-1,
    \]
    \item stability filters satisfied:
    \[
    \mu(t) > \mu_{\min}(W(t)),\qquad
    |C(t)| > \alpha M(t),\qquad
    \Theta(t) < \Theta_{\max}.
    \]
\end{enumerate}

If FEW conditions hold:
\[
\text{FTT\_API}(t) =
\begin{cases}
1, & K(t)=+1,\\[4pt]
2, & K(t)=-1.
\end{cases}
\]

\subsection{NONE Logic}

If neither FEW nor FXB conditions are satisfied:
\[
\text{FTT\_API}(t) = 0.
\]

NONE indicates that the structure remains either linear, transitional, or
propagating without collapse.

\subsection{Determinism and Purity}

FTT\_API satisfies the following formal guarantees:

\begin{itemize}[noitemsep]
    \item \textbf{Determinism:} identical price sequences produce identical outputs.
    \item \textbf{Locality:} decisions depend only on recent values within $W(t)$.
    \item \textbf{Purity:} no stochastic behavior, smoothing, or empirical tuning.
    \item \textbf{Completeness:} every structural state maps to exactly one output.
\end{itemize}

Thus, FTT\_API defines a closed, fully deterministic geometric decision system.

\section{Mapping to Runtime Engine (FTRE)}
\label{sec:ftre}

The Flexion Trading Runtime Engine (FTRE) is the canonical executable
implementation of Flexion Trading Theory.  
Its purpose is to translate the mathematical constructs of FTT into a
deterministic, platform-independent computational process.  
FTRE introduces no new logic and uses no additional parameters; every step is
a direct operationalization of the theoretical rules defined in
Sections~\ref{sec:flexion-derivatives}–\ref{sec:api}.

FTRE ensures that FTT remains:
\begin{itemize}[noitemsep]
    \item reproducible across environments,
    \item invariant under platform-specific differences,
    \item free from heuristics, smoothing, or stochastic behavior,
    \item strictly deterministic given identical price sequences.
\end{itemize}

\subsection{Architectural Principles}

The runtime engine adheres to three non-negotiable invariants:

\paragraph{Purity.}
All structural quantities computed inside FTRE
\[
\Delta P(t),\;
C(t),\;
\Delta S(t),\;
K(t),\;
W(t),\;
\mu(t),\;
M(t),\;
\Theta(t)
\]
must be derived exclusively from $P(t)$ and its discrete differences.

\paragraph{Determinism.}
For any two identical price sequences, FTRE must produce identical sequences of
structural events and API outputs.  
No randomness, smoothing, or adaptive heuristics may influence decision
generation.

\paragraph{Canonical Mapping.}
FTRE must implement FTT\_API exactly as defined in
Section~\ref{sec:api}, without alteration or extension of event logic.

\subsection{Internal Processing Pipeline}

For each new observed price $P(t)$, FTRE executes the following deterministic
pipeline:

\begin{enumerate}[label=\textbf{Step \arabic*:}, leftmargin=2cm]

    \item \textbf{Window Update}  
    Append $P(t)$ to local memory; adapt $W(t)$ via curvature stability;  
    discard values outside the structural window.

    \item \textbf{Compute Flexion Derivatives}  
    \[
    \Delta P(t) = P(t) - P(t-1),
    \]
    \[
    C(t) = \Delta P(t) - \Delta P(t-1),
    \]
    \[
    \Delta S(t) = C(t) - C(t-1).
    \]

    \item \textbf{Compute Structural Orientation}  
    \[
    K(t) = \mathrm{sign}\!\left( \sum_{i=0}^{W(t)} C(t-i) \right).
    \]

    \item \textbf{Compute Stability Filters}  
    \[
    \mu(t) = |\Delta P(t)|,
    \]
    \[
    M(t) = \frac{1}{W(t)+1}\sum_{i=0}^{W(t)} |C(t-i)|,
    \]
    \[
    \Theta(t) = \frac{W(t)}{|\Delta S(t)| + \varepsilon }.
    \]

    \item \textbf{Event Evaluation}  
    Evaluate FXB and FEW using the strict priority order defined in
    Section~\ref{sec:api}.

    \item \textbf{State Transition}  
    Update the structural trading state (FLAT, LONG, SHORT) using deterministic
    transition rules.

    \item \textbf{Output Structural Code}  
    Return the integer action code:
    \[
    0 = \text{NONE},\quad
    1 = \text{FEW\_UP},\quad
    2 = \text{FEW\_DN},\quad
    3 = \text{FXB}.
    \]

\end{enumerate}

This pipeline contains no conditional branches beyond those defined by the
theory.  
All computations involve only primitive arithmetic operations and minimal
memory, ensuring high performance and platform independence.

\subsection{Canonical Public API}

FTRE exposes exactly one public method:

\[
\texttt{int FTRE\_Evaluate(double price, int timestamp);}
\]

which returns:
\[
0,\;1,\;2,\;3
\]
according to the structural state at time $t$.  
No additional outputs, diagnostic flags, or auxiliary data streams are allowed
at the API level.  
Any debugging or logging functionality must remain strictly optional and
external.

\subsection{Implementation Independence}

FTRE must produce identical results across:

\begin{itemize}[noitemsep]
    \item programming languages (C++, Python, MQL4/5),
    \item operating systems (Linux, Windows, macOS),
    \item brokers and trading platforms,
    \item live and historical price feeds,
    \item varying tick arrival rates or latencies.
\end{itemize}

Platform differences must not influence structural outcomes because FTT
depends only on the ordering and values of $P(t)$.

\subsection{Completeness of Mapping}

Every component of FTRE corresponds directly to a theoretical construct:

\[
\begin{aligned}
\text{Derivative Engine} &\;\longleftrightarrow\; \text{Section~3}, \\
\text{Orientation Engine} &\;\longleftrightarrow\; \text{Section~4}, \\
\text{Event Engine} &\;\longleftrightarrow\; \text{Section~5}, \\
\text{Filter Engine} &\;\longleftrightarrow\; \text{Section~6}, \\
\text{Cycle Logic} &\;\longleftrightarrow\; \text{Section~7}, \\
\text{Decision Function} &\;\longleftrightarrow\; \text{Section~8}. \\
\end{aligned}
\]

No additional logic exists beyond the theory.  
FTRE is therefore the authoritative executable reference of Flexion Trading
Theory.

\section{Discussion}
\label{sec:discussion}

Flexion Trading Theory introduces a novel geometric framework for analyzing
market dynamics, diverging from both traditional technical analysis and
statistical modelling paradigms.  
Rather than relying on volatility estimates, oscillators, or probabilistic
assumptions, FTT extracts structure directly from the discrete evolution of
price through flexion derivatives.  
This perspective yields several important theoretical and practical
implications.

\subsection{A Deterministic Alternative to Classical Approaches}

Classical trading methodologies often incorporate:
\begin{itemize}[noitemsep]
    \item smoothed indicators,
    \item stochastic models,
    \item probabilistic forecasts,
    \item heuristic rules or empirical tuning.
\end{itemize}

These approaches introduce inherent uncertainty and platform dependence.
In contrast, FTT is entirely deterministic:  
given identical price sequences, it produces identical decisions across all
environments.  
This determinism eliminates ambiguity and enables reproducibility across
implementations, making FTT suitable for rigorous scientific and engineering
contexts.

\subsection{Structural Interpretation of Market Impulses}

By interpreting price evolution as a geometric object, FTT reframes market
impulses as structural transitions:
\begin{itemize}[noitemsep]
    \item FEW corresponds to the birth of curvature (structural initiation),
    \item FXB corresponds to torsional extremum (structural collapse).
\end{itemize}

This geometric viewpoint provides a coherent explanation for market movements
without appealing to:
\begin{itemize}[noitemsep]
    \item behavioral assumptions,
    \item statistical noise models,
    \item pattern recognition templates.
\end{itemize}

Flexion structure becomes the lens through which market motion is organized and
understood.

\subsection{Role of Adaptive Windows and Filters}

The adaptive structural window $W(t)$ and the stability filters
$\mu(t)$, $M(t)$, and $\Theta(t)$ serve an essential purpose:  
they preserve the geometric purity of flexion derivatives while ensuring that
structural events reflect meaningful market behavior rather than
microstructural noise.

FTT avoids the pitfalls of both extremes:
\begin{itemize}[noitemsep]
    \item it does not smooth or distort the underlying data,
    \item yet it prevents overreaction to insignificant fluctuations.
\end{itemize}

This balance enables FTT to operate effectively on high-frequency tick data
while maintaining a principled theoretical foundation.

\subsection{Universality and Platform Independence}

Because FTT depends solely on the ordering and values of $P(t)$, and not on
wall-clock time or platform-specific features, it remains invariant across:
\begin{itemize}[noitemsep]
    \item brokers,
    \item executions speeds,
    \item feed irregularities,
    \item historical and live data environments.
\end{itemize}

This universality is especially important for reproducible research and for
deploying FTRE across heterogeneous computational infrastructures.

\subsection{Limitations and Scope}

FTT does not attempt to model:
\begin{itemize}[noitemsep]
    \item long-term price trends,
    \item macroeconomic factors,
    \item liquidity conditions,
    \item spread, slippage, or execution constraints.
\end{itemize}

Its scope is purely structural:  
FTT describes \emph{when} bending begins and \emph{when} it collapses.

Execution-layer considerations, portfolio interactions, risk management, and
market frictions lie outside the theory itself but can be integrated in
systems that consume FTT\_API outputs.

\subsection{Implications for Algorithmic Trading}

The strict determinism, minimalism, and universality of FTT make it well suited
for:
\begin{itemize}[noitemsep]
    \item automated strategy design,
    \item real-time execution engines,
    \item scientific backtesting,
    \item reproducible performance comparisons.
\end{itemize}

Because FTRE can be implemented as a platform-independent library, FTT can
serve as a stable mathematical core in high-frequency environments, research
platforms, and production trading systems.

\section{Conclusion}
\label{sec:conclusion}

Flexion Trading Theory provides a deterministic and structurally grounded
framework for interpreting market dynamics through the geometry of price
evolution.  
By deriving curvature, torsion, and orientation directly from the discrete
sequence of observable prices, FTT avoids reliance on stochastic assumptions,
parameter tuning, or traditional technical indicators.  
Instead, market behavior is described in terms of coherent bending regimes and
their eventual collapse.

The theory defines a complete structural lifecycle through two fundamental
events:  
\begin{itemize}[noitemsep]
    \item \textbf{FEW} --- initiation of a bending regime,  
    \item \textbf{FXB} --- torsional collapse and termination.
\end{itemize}

These events, combined with adaptive windows and purely geometric stability
filters, yield a formal decision function that maps raw price data to trading
signals with full determinism and platform independence.

Furthermore, the theory maps directly and unambiguously to the Flexion Trading
Runtime Engine (FTRE), ensuring consistent and faithful implementation across
all environments.  
This separation between theory and execution preserves scientific rigor while
enabling robust engineering applications.

FTT thus establishes a unified geometric viewpoint on short-term market
structure, offering a reproducible alternative to probabilistic or heuristic
approaches.  
Its minimal assumptions, strict locality, and computational clarity provide a
solid theoretical foundation for future developments in algorithmic trading,
structural market research, and automated decision systems.

\end{document}
