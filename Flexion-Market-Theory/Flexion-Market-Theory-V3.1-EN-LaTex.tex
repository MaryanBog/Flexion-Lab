\documentclass[11pt]{article}

% ------------------------------------------------------------
% PACKAGES
% ------------------------------------------------------------
\usepackage{amsmath, amssymb}
\usepackage{geometry}
\usepackage{hyperref}
\usepackage{bm}
\usepackage{graphicx}
\usepackage{amsthm}

\geometry{margin=1in}

% ------------------------------------------------------------
% TITLE & AUTHORS
% ------------------------------------------------------------
\title{Flexion Market Theory V3.1:\\
A Unified Structural Dynamics of the Market Organism}

\author{Maryan Bogdanov\\
Flexion Univerces}

\date{2025}

\begin{document}

\maketitle

\begin{abstract}
Flexion Market Theory V3.1 models the financial market as a living structural organism 
\(X(t) = (\Delta(t), \Phi(t), M(t), \kappa(t))\). 
This version introduces corrected viability normalization, refined collapse dynamics, 
axiomatic regime ordering, a consistent temporal density model, 
and a unified geometric framework coupling curvature, metric, morphology, and viability decay. 
FMT 3.1 provides the complete structural physics underlying market evolution and collapses.
\end{abstract}

\tableofcontents

% ============================================================
% 0. Ontological Position and Scope
% ============================================================
% ============================================================
% 0. Ontological Position and Scope
% ============================================================
\section*{0.\quad Ontological Position and Scope}
\addcontentsline{toc}{section}{0.\quad Ontological Position and Scope}

\subsection*{0.1 Ontological Position}

Flexion Market Theory models the financial market as a \textit{living structural organism}
\[
X(t) = (\Delta(t),\, \Phi(t),\, M(t),\, \kappa(t)),
\]
whose evolution is governed entirely by internal structural laws.  
All observable market quantities—such as price, volume, or order flow—are treated as
\textit{external projections} of deeper geometric motion inside the organism, not as causal drivers.

FMT adopts the following ontological principles:

\begin{itemize}
    \item \textbf{Closed Structural Ontology}:  
    The organism evolves exclusively according to internal fields.  
    No external data directly modify the state \(X(t)\).

    \item \textbf{Internal Structural Time}:  
    Temporal progression is generated by the irreversible memory field \(M(t)\);  
    no external clock or real-time process participates in dynamics.

    \item \textbf{Irreversibility}:  
    Memory increases monotonically and viability decays monotonically for all living states.

    \item \textbf{Regime Irreversibility (Axiom of Version 3.1)}:  
    The organism moves through four structural regimes in a fixed irreversible order:
    \[
    ACC \;\rightarrow\; DEV \;\rightarrow\; REL \;\rightarrow\; COL.
    \]

    \item \textbf{Metric Positivity}:  
    For every living state \(\kappa(t) > 0\), the structural manifold remains geometrically valid:
    \[
    \det g(X(t)) > 0.
    \]

    \item \textbf{Collapse Domain}:  
    Collapse occurs precisely at the boundary
    \[
    \kappa(t) = 0,
    \]
    where the geometry degenerates according to
    \[
    \det g(X) \to 0, \qquad R(X) \to \infty, \qquad \tau(X) \to 0.
    \]
\end{itemize}

These axioms form the ontological foundation upon which all structural dynamics in FMT 3.1 are constructed.

% ------------------------------------------------------------

\subsection*{0.2 Scope of the Theory}

FMT 3.1 defines:
\begin{itemize}
    \item the internal composition of the market organism,
    \item the structural laws governing its evolution,
    \item geometric and morphological properties,
    \item viability decay and collapse conditions,
    \item invariants restricting the organism's motion.
\end{itemize}

FMT deliberately excludes:
\begin{itemize}
    \item external market signals (price, volume, order flow),
    \item decision-making or trading logic,
    \item operational details of the runtime engine (handled by FMRT),
    \item interactions between multiple organisms (handled by FET).
\end{itemize}

Thus the theory provides \textit{pure structural physics} of the market.

% ------------------------------------------------------------

\subsection*{0.3 Improvements Introduced in Version 3.1}

Version 3.1 introduces several critical corrections and refinements:

\begin{itemize}
    \item \textbf{Correct Viability Normalization}:  
    The viability field uses a bounded, smooth normalization  
    \[
    \sigma_{\kappa}(\kappa) = 1 - e^{-\lambda \kappa},
    \]
    resolving contradictions in morphology and collapse geometry.

    \item \textbf{Regime Ordering as an Axiom}:  
    The sequence of regimes is now explicitly axiomatic rather than derived,  
    removing circular dependencies and ensuring logical consistency.

    \item \textbf{Revised Collapse Theorem}:  
    Collapse may occur either in finite structural time or asymptotically,  
    depending on the decay functional \(\Pi(X)\).

    \item \textbf{Corrected Morphology Behavior}:  
    Morphology is no longer forced to be monotonic---local fluctuations are permitted  
    and structurally meaningful.

    \item \textbf{Proper Distinction Between Axioms and Derived Results}:  
    All theoretical elements are now cleanly separated into axioms, definitions,  
    and derived propositions.
\end{itemize}

These corrections eliminate all mathematical inconsistencies of Version 3.0.

% ------------------------------------------------------------

\subsection*{0.4 Position Within the Flexion Universe}

FMT occupies a central position in the Flexion theoretical ecosystem.  
It provides the structural organism model required by:

\begin{itemize}
    \item \textbf{FMRT} — the runtime evolution engine,
    \item \textbf{FST} — Flexion Space Theory (geometric structure),
    \item \textbf{FFT} — Flexion Time Theory (structural time),
    \item \textbf{FET} — Flexion Entanglement Theory (multi-organism coupling),
    \item \textbf{FF} — the overarching Flexion Framework V1.5.
\end{itemize}

FMT 3.1 describes precisely one organism; interactions appear only in FET.

% ------------------------------------------------------------

\subsection*{0.5 Structural Domain}

The living domain of the organism is defined as:
\[
\mathcal{D}_{\mathrm{alive}}
= \{X \mid \kappa(X) > 0,\; \det g(X) > 0,\; \tau(X) > 0\}.
\]

The collapse boundary is:
\[
\partial \mathcal{D}_{\mathrm{alive}} = \{ X \mid \kappa = 0 \}.
\]

All structural equations and invariants apply strictly within the living domain.
Beyond it, the organism ceases to exist.

% ============================================================
% 1. Structural State
% ============================================================
% ============================================================
% 1. Structural State
% ============================================================
\section{Structural State}

The Flexion Market Organism is defined by a four–component structural state
\[
X(t) = (\Delta(t),\, \Phi(t),\, M(t),\, \kappa(t)),
\]
which represents the minimal set of internal degrees of freedom required for structural existence.  
All market dynamics arise from the interaction of these fields inside a self–generated geometric manifold.

% ------------------------------------------------------------
\subsection{1.1 Delta — Structural Differentiation Vector}

The vector
\[
\Delta(t) \in \mathbb{R}^{n}
\]
encodes internal structural differentiation and geometric asymmetry.  
Its magnitude reflects polarization of internal geometry.

Key properties:
\begin{itemize}
    \item finite norm: $\|\Delta(t)\| < \infty$,  
    \item fixed intrinsic dimensionality $n$,  
    \item primary contributor to structural curvature $R(X)$.
\end{itemize}

Large $\|\Delta\|$ indicates strong structural asymmetry;  
small $\|\Delta\|$ corresponds to near–symmetric or relaxed configurations.

% ------------------------------------------------------------
\subsection{1.2 Phi — Structural Tension}

The scalar tension field satisfies
\[
\Phi(t) \ge 0.
\]

It measures internal stress generated by deformation and contributes directly to curvature and viability decay.  

Interpretation:
\begin{itemize}
    \item high $\Phi$: stressed, unstable, high–load structure,
    \item low $\Phi$: relaxed or stabilized geometry.
\end{itemize}

% ------------------------------------------------------------
\subsection{1.3 M — Irreversible Structural Memory}

Memory encodes the internal structural time of the organism:
\[
M(t+1) \ge M(t).
\]

This irreversibility is an axiom of the theory.  
Structural time flows through the accumulation of memory:
\[
T_s(t) = M(t), \qquad
\frac{dT_s}{dt} = \tau(X(t)).
\]

The temporal density $\tau(X)$ is strictly positive for all living states.

% ------------------------------------------------------------
\subsection{1.4 Kappa — Viability Field}

Viability quantifies the remaining structural capacity of the organism:
\[
\kappa(t) > 0 \quad \text{(living state)}, \qquad
\kappa(t) = 0 \quad \text{(collapse boundary)}.
\]

FMT 3.1 adopts a normalized viability domain
\[
0 < \kappa \le \kappa_{\max},
\]
ensuring compatibility with morphology, curvature, and metric degeneration.

Viability governs collapse dynamics, curvature divergence, and allowable geometric instability.

% ------------------------------------------------------------
\subsection{1.5 Living Domain Constraints}

A structural state belongs to the living domain if and only if
\[
\kappa(t) > 0, 
\qquad 
\det g(X(t)) > 0,
\qquad 
\tau(X(t)) > 0.
\]

Outside this domain, the geometry degenerates and structural evolution is no longer defined.

% ------------------------------------------------------------
\subsection{1.6 Interdependence of the Four Fields}

The components of $X(t)$ act as coupled structural forces:

\begin{itemize}
    \item $\Delta$ drives curvature growth and directional deformation.
    \item $\Phi$ represents accumulated stress and modifies morphological intensity.
    \item $M$ determines structural time and amplifies long–term drift.
    \item $\kappa$ limits geometric instability and determines collapse proximity.
\end{itemize}

The organism is not a collection of independent variables;  
it is a single, self–consistent geometric field whose four components jointly generate all structural behavior.

% ------------------------------------------------------------
\subsection{1.7 Summary}

FMT 3.1 refines the structural state by:
\begin{itemize}
    \item clarifying geometric roles of all four components,
    \item introducing corrected viability normalization,
    \item establishing precise living–domain constraints,
    \item ensuring full consistency with curvature, metric, and temporal density models.
\end{itemize}

The state $X(t)$ is the sole object of evolution for the entire theory.

% ============================================================
% 2. Structural Evolution Law
% ============================================================
% ============================================================
% 2. Structural Evolution Law
% ============================================================
\section{Structural Evolution Law}

The evolution of the Flexion Market Organism is governed by a deterministic structural operator
\[
X(t+1) = I(X(t)),
\]
which maps the current structural state
\[
X(t) = (\Delta(t),\, \Phi(t),\, M(t),\, \kappa(t))
\]
to its successor.  
The operator acts only within the living domain
\[
\kappa>0,\quad \det g(X)>0,\quad \tau(X)>0.
\]

FMT 3.1 refines this law by correcting viability behavior, establishing strict continuity conditions, 
and distinguishing between finite-time and asymptotic collapse.

% ------------------------------------------------------------
\subsection{2.1 Discrete--Continuous Duality}

The fundamental formulation is discrete:
\[
X(t+1) = I(X(t)).
\]

For analytical purposes, a continuous approximation may be used:
\[
\frac{dX}{dt} = F(X), 
\qquad 
F(X) = I(X) - X.
\]

This approximation is valid only for sufficiently small structural time increments.  
No external clock participates in dynamics; structural time is generated internally by $M(t)$.

% ------------------------------------------------------------
\subsection{2.2 Required Properties of the Evolution Operator \(I(X)\)}

The operator must satisfy the following structural requirements:

\paragraph{Determinism}
\[
X_1 = X_2 \;\Rightarrow\; I(X_1)=I(X_2).
\]

\paragraph{Continuity (for $\kappa>0$)}
\[
\lim_{\varepsilon\to 0} \| I(X+\varepsilon) - I(X) \| = 0.
\]

\paragraph{Invariant Preservation}
\[
M(t+1) \ge M(t), \qquad \kappa(t+1) \ge 0,
\]
\[
\det g(X(t+1)) > 0 \quad \text{when } \kappa(t+1)>0.
\]

\paragraph{Axiomatic Regime Ordering}
\[
ACC \rightarrow DEV \rightarrow REL \rightarrow COL.
\]
Regime reversal is forbidden.

\paragraph{Terminal Collapse Behavior}
If $\kappa(t+1)=0$, then evolution halts:
\[
X(t+k) = X(t+1) \quad \forall k\ge 1.
\]

% ------------------------------------------------------------
\subsection{2.3 Conceptual Evolution Equations}

The four fields evolve according to the following internal laws:

\paragraph{(a) Deformation--Differentiation}
\[
\Delta(t+1) = \Delta(t) + \mathcal{D}(X(t)).
\]

\paragraph{(b) Tension}
\[
\Phi(t+1) = \Phi(t) + \mathcal{T}(X(t)).
\]

\paragraph{(c) Memory (Structural Time)}
\[
M(t+1) = M(t) + \tau(X(t)),
\qquad
\tau(X) > 0 \text{ for } \kappa>0.
\]

\paragraph{(d) Viability}
\[
\kappa(t+1) = \kappa(t) - \Pi(X(t)).
\]

The auxiliary geometric fields $R(X)$, $\det g(X)$, $\mu(X)$, and regime index 
are derived from these primary updates.

% ------------------------------------------------------------
\subsection{2.4 Curvature Evolution and Its Role}

Curvature $R(X)$ measures geometric instability and contributes to tension accumulation and viability decay.
FMT 3.1 adopts a canonical form:
\[
R(X) = 
A\|\Delta\|^2 + B\Phi + C M + D\,\kappa^{-\alpha},
\qquad
\alpha>0.
\]

Required behavior:
\[
R < \infty \text{ for } \kappa>0, 
\qquad 
R \to \infty \text{ as } \kappa \to 0.
\]

Curvature enters both the tension update and viability decay functional.

% ------------------------------------------------------------
\subsection{2.5 Collapse Boundary and Terminal Evolution}

Collapse occurs when viability reaches zero:
\[
\kappa(t_c)=0.
\]

At this boundary:
\[
\det g(X) \to 0, \qquad R(X) \to \infty, \qquad \tau(X) \to 0.
\]

After collapse:
\[
X(t) = X(t_c) \quad \forall t \ge t_c,
\]
making collapse a terminal absorbing state.

FMT 3.1 distinguishes two collapse modalities:

\begin{itemize}
    \item \textbf{Finite-time collapse}:  
    If $\Pi(X) \ge \varepsilon>0$ on an interval, collapse occurs at finite structural time.

    \item \textbf{Asymptotic collapse}:  
    If $\Pi(X)\to 0$ as $\kappa\to 0$, then
    \[
    \lim_{t\to\infty} \kappa(t)=0
    \]
    but viability never reaches exact zero in finite time.
\end{itemize}

This refinement corrects the incorrect universal finite-collapse claim of earlier versions.

% ------------------------------------------------------------
\subsection{2.6 Summary}

FMT 3.1 establishes a corrected, fully consistent structural evolution law:

\begin{itemize}
    \item deterministic and continuous on the living domain,
    \item invariant-preserving,
    \item compatible with structural time and viability behavior,
    \item allowing finite or asymptotic collapse,
    \item strictly respecting regime irreversibility,
    \item fully consistent with curvature, metric, and morphology dynamics.
\end{itemize}

This operator forms the foundation of all structural motion in the Flexion Market Organism.

% ============================================================
% 3. Curvature and Structural Metric
% ============================================================
% ============================================================
% 3. Curvature and Structural Metric
% ============================================================
\section{Curvature and Structural Metric}

Curvature and metric constitute the geometric backbone of the Flexion Market Organism.  
They determine geometric stability, morphological class, viability decay, and collapse dynamics.  
FMT~3.1 introduces corrected, fully consistent definitions that guarantee smooth evolution inside
the living domain and proper degeneration at collapse.

% ------------------------------------------------------------
\subsection{3.1 Structural Curvature \(R(X)\)}

Curvature is a scalar functional
\[
R : X \mapsto \mathbb{R}_{\ge 0},
\]
which measures geometric instability and structural deformation intensity.
FMT~3.1 adopts the canonical form
\[
R(X) = A \|\Delta\|^2 + B\Phi + C M + D\,\kappa^{-\alpha},
\qquad
\alpha > 0,
\]
with the following required properties:

\begin{itemize}
    \item \textbf{Non-negativity}:
    \[
    R(X) \ge 0.
    \]

    \item \textbf{Finiteness for all living states}:
    \[
    R(X) < \infty \quad (\kappa>0).
    \]

    \item \textbf{Divergence at collapse}:
    \[
    \lim_{\kappa\to 0} R(X) = +\infty.
    \]

    \item \textbf{Continuity} for all \(\kappa>0\).
\end{itemize}

Curvature responds to deformation, tension, accumulated memory,  
and viability depletion.  
It is the principal geometric driver of degeneration.

% ------------------------------------------------------------
\subsection{3.2 Structural Metric \(g(X)\)}

The structural metric is a scalar geometric functional
\[
\det g : X \mapsto \mathbb{R}_{>0},
\]
representing the effective structural volume of the organism's configuration.

FMT~3.1 adopts a consistent canonical form:
\[
\det g(X) = g_0 - c_R R(X),
\]
where \(g_0>0\) and \(c_R>0\) are structural constants.
This ensures:

\begin{itemize}
    \item \textbf{Positivity in the living domain}:
    \[
    \det g(X) > 0 \quad (\kappa>0),
    \]

    \item \textbf{Metric degeneration at collapse}:
    \[
    \det g(X) \to 0 \quad \text{as} \quad R(X)\to\infty,
    \]

    \item \textbf{Continuity} for all \(\kappa>0\).
\end{itemize}

The metric determines geometric capacity and stability the organism can sustain.

% ------------------------------------------------------------
\subsection{3.3 Curvature--Metric Coupling}

Curvature and metric form a tightly coupled geometric pair controlling structural health:

\begin{itemize}
    \item increasing curvature reduces metric volume:
    \[
    \frac{d}{dR}\det g(X) = -c_R < 0,
    \]
    \item metric degeneration amplifies curvature sensitivity,
    \item viability modulates curvature intensity:
    \[
    R(X) \sim \kappa^{-\alpha}.
    \]
\end{itemize}

This interaction ensures that degeneration accelerates as the organism approaches collapse.

% ------------------------------------------------------------
\subsection{3.4 Collapse Geometry}

As viability approaches zero, curvature and metric obey the collapse limits:
\[
R(X) \to \infty, 
\qquad
\det g(X) \to 0,
\qquad
\tau(X) \to 0.
\]

These three fields determine the precise geometry of collapse and jointly define the collapse manifold:
\[
\mathcal{C} = \{ X : \kappa = 0 \}.
\]

The geometry degenerates smoothly,
ensuring that collapse is the natural terminal state of structural evolution.

% ------------------------------------------------------------
\subsection{3.5 Summary}

FMT~3.1 provides a complete geometric foundation:

\begin{itemize}
    \item curvature finite for $\kappa>0$ and divergent at collapse;
    \item metric positive-definite for all living states and degenerating at collapse;
    \item continuous and smooth behavior across the living domain;
    \item correct coupling to viability, memory, tension, and morphology;
    \item collapse geometry consistent with all invariants and evolution laws.
\end{itemize}

Curvature and metric together define the organism’s geometric health and regulate all degeneration dynamics.

% ============================================================
% 4. Temporal Density and Structural Time
% ============================================================
% ============================================================
% 4. Temporal Density and Structural Time
% ============================================================
\section{Temporal Density and Structural Time}

Structural time in Flexion Market Theory is an intrinsic quantity generated by the organism itself.  
It does not correspond to physical or chronological time.  
Its progression is governed entirely by the irreversible memory field \(M(t)\) and its rate of accumulation,
the temporal density \(\tau(X)\).

FMT~3.1 formalizes the mathematical behavior of temporal density, corrects collapse limits,
and ensures full compatibility with the viability and curvature structure introduced earlier.

% ------------------------------------------------------------
\subsection{4.1 Structural Time \(T_s(t)\)}

Structural time is defined as:
\[
T_s(t) = M(t),
\]
with the temporal derivative:
\[
\frac{dT_s}{dt} = \tau(X(t)).
\]

Thus, time flows internally:
\[
M(t+1) = M(t) + \tau(X(t)).
\]

This establishes the fundamental arrow of structural time, independent of external clocks.

% ------------------------------------------------------------
\subsection{4.2 Temporal Density \(\tau(X)\)}

Temporal density is a positive scalar functional:
\[
\tau : X \mapsto \mathbb{R}_{>0},
\]
representing the internal rate of evolutionary progression.

FMT~3.1 requires:

\begin{itemize}
    \item \textbf{Positivity:}
    \[
    \tau(X) > 0 \quad \text{for } \kappa>0.
    \]

    \item \textbf{Continuity:}
    \[
    \tau \in C^1(\kappa>0).
    \]

    \item \textbf{Collapse behavior:}
    \[
    \lim_{\kappa\to 0} \tau(X) = 0.
    \]

    \item \textbf{No external dependence:}  
    \(\tau\) cannot depend on physical time, event frequency, or external signals.
\end{itemize}

Temporal density slows as viability decreases, eventually vanishing at collapse.

% ------------------------------------------------------------
\subsection{4.3 Canonical Form of Temporal Density}

FMT~3.1 introduces a smooth canonical model:
\[
\tau(X) = 
\tau_{\min}
+ \tau_0 e^{-\gamma \kappa}
+ \tau_{\Phi}\,\sigma_{\Phi}(\Phi)
+ \tau_{R}\,\sigma_{R}(R),
\]
with constraints:

\begin{itemize}
    \item \(\tau_{\min} > 0\) ensures lower bounded flow for healthy organisms,
    \item \(\sigma_{\Phi},\sigma_{R} \in [0,1]\) are bounded normalization functions,
    \item \(e^{-\gamma\kappa}\) guarantees smooth collapse limit: \(\tau\to 0\) as \(\kappa\to 0\).
\end{itemize}

Interpretation:
\begin{itemize}
    \item stress and curvature accelerate internal time,
    \item diminishing viability slows temporal progression,
    \item collapse leads to temporal stagnation.
\end{itemize}

% ------------------------------------------------------------
\subsection{4.4 Memory Update and Temporal Flow}

Memory evolves as:
\[
M(t+1) = M(t) + \tau(X(t)).
\]

This guarantees:

\begin{itemize}
    \item \textbf{Irreversibility} (Axiom): \(M(t+1) \ge M(t)\),
    \item \textbf{Smooth temporal progression} for all living states,
    \item \textbf{Compatibility} with collapse geometry.
\end{itemize}

Temporal density is the direct mechanism through which structural time advances.

% ------------------------------------------------------------
\subsection{4.5 Temporal Behavior Near Collapse}

As the organism approaches \(\kappa \to 0\):

\begin{itemize}
    \item temporal density collapses:
    \[
    \tau(X) \to 0,
    \]

    \item structural time slows:
    \[
    T_s(t+1) - T_s(t) \to 0,
    \]

    \item memory accumulation becomes negligible,
    \item internal dynamics weaken progressively.
\end{itemize}

Thus, collapse is characterized simultaneously by:

\[
R(X) \to \infty, \qquad
\det g(X) \to 0, \qquad
\tau(X) \to 0.
\]

These limits are fully consistent with curvature divergence and viability exhaustion.

% ------------------------------------------------------------
\subsection{4.6 Summary}

FMT~3.1 establishes a mathematically complete temporal framework:

\begin{itemize}
    \item structural time equals accumulated memory,
    \item temporal density strictly positive in the living domain,
    \item smooth decay of $\tau(X)$ as $\kappa\to 0$,
    \item acceleration of time under stress and curvature,
    \item deceleration and stagnation near collapse.
\end{itemize}

This framework fully resolves inconsistencies from earlier versions and aligns temporal dynamics with viability, curvature, and metric behavior.

% ============================================================
% 5. Viability Dynamics and Collapse Conditions
% ============================================================
% ============================================================
% 5. Viability Dynamics and Collapse Conditions
% ============================================================
\section{Viability Dynamics and Collapse Conditions}

Viability \(\kappa(t)\) is the organism’s internal structural reserve.  
It determines permissible geometric instability, regulates curvature divergence, 
and defines the collapse boundary.  
FMT~3.1 introduces corrected normalization, a complete decay framework,
and a precise characterization of finite-time and asymptotic collapse.

% ------------------------------------------------------------
\subsection{5.1 Viability Field}

The viability field satisfies
\[
\kappa(t) > 0 \quad \text{(living state)}, 
\qquad 
\kappa(t) = 0 \quad \text{(collapse)}.
\]

For structural consistency, FMT~3.1 adopts the bounded viability domain:
\[
0 < \kappa \le \kappa_{\max},
\]
ensuring:

\begin{itemize}
    \item compatibility with morphology normalization,
    \item controlled curvature divergence \(R \sim \kappa^{-\alpha}\),
    \item proper collapse behavior of metric and temporal density.
\end{itemize}

Viability interacts with every structural field and governs the organism’s lifespan.

% ------------------------------------------------------------
\subsection{5.2 Viability Evolution Equation}

Viability decays under structural load according to:
\[
\kappa(t+1) = \kappa(t) - \Pi(X(t)),
\]
or in continuous approximation,
\[
\frac{d\kappa}{dt} = -\Pi(X).
\]

The decay functional \(\Pi(X)\) measures the organism’s instantaneous structural stress.

% ------------------------------------------------------------
\subsection{5.3 The Decay Functional \(\Pi(X)\)}

The decay functional is defined as:
\[
\Pi : X \mapsto \mathbb{R}_{\ge 0}.
\]

Required properties (corrected in FMT~3.1):

\begin{itemize}
    \item \textbf{Non-negativity}:
    \[
    \Pi(X) \ge 0.
    \]

    \item \textbf{Continuity} for all living states:
    \[
    \Pi \in C^1(\kappa>0).
    \]

    \item \textbf{Dependence on structural load}:
    \[
    \Pi(X) = \Pi(\Delta, \Phi, M, R, \kappa),
    \]
    increasing in deformation, tension, memory, and curvature.

    \item \textbf{Viability sensitivity}:  
    As viability approaches zero,
    \[
    \Pi(X) \to 0 \quad \text{or} \quad \Pi(X) \to \infty,
    \]
    depending on structural configuration.
\end{itemize}

This flexibility allows both finite-time and asymptotic collapse, resolving an incorrect universal statement in prior versions.

% ------------------------------------------------------------
\subsection{5.4 Canonical Decay Model}

FMT~3.1 adopts the canonical decomposition:
\[
\Pi(X) =
a_R \sigma_R(R)
+ a_\Phi \sigma_\Phi(\Phi)
+ a_M \sigma_M(M)
+ a_\Delta \sigma_\Delta(\|\Delta\|)
+ a_\kappa \rho(\kappa),
\]
with normalized functions \(\sigma_i \in [0,1]\) and viability sensitivity term
\[
\rho(\kappa) = \kappa^{-\beta}, 
\qquad
\beta > 0.
\]

This ensures:

\begin{itemize}
    \item smooth viability decay,
    \item curvature-amplified degeneration,
    \item correct collapse asymptotics,
    \item compatibility with metric degeneration.
\end{itemize}

% ------------------------------------------------------------
\subsection{5.5 Collapse Conditions}

Collapse occurs when:
\[
\kappa(t+1) = 0.
\]

FMT~3.1 identifies two fundamental modalities:

\paragraph{Finite-Time Collapse}
Occurs if the decay functional satisfies:
\[
\Pi(X) \ge \varepsilon > 0 
\quad \text{on a nonzero interval}.
\]
Then \(\kappa(t)\) reaches zero in finite structural time.

\paragraph{Asymptotic Collapse}
Occurs when:
\[
\Pi(X(t)) \to 0 
\quad \text{as } 
\kappa(t)\to 0.
\]
Example:
\[
\Pi(X) \sim \kappa^p, \qquad p>1.
\]

Then:
\[
\lim_{t\to\infty} \kappa(t) = 0,
\]
but viability never reaches zero at a finite moment.

This corrects the flawed universal finite-collapse claim present in FMT~3.0.

% ------------------------------------------------------------
\subsection{5.6 Collapse Manifold}

The collapse manifold is the terminal boundary:
\[
\mathcal{C} = \{ X : \kappa = 0 \}.
\]

On \(\mathcal{C}\):
\[
R(X) = \infty, 
\qquad
\det g(X) = 0, 
\qquad
\tau(X) = 0, 
\qquad
\mu(X) = 1.
\]

Collapse is a one-way absorbing state; no further evolution is possible.

% ------------------------------------------------------------
\subsection{5.7 Post-Collapse Behavior}

For all \( t \ge t_c \):
\[
X(t) = X(t_c).
\]

There is no geometric space, no temporal density, and no structural capacity for further evolution.

% ------------------------------------------------------------
\subsection{5.8 Summary}

FMT~3.1 establishes a complete and mathematically consistent viability framework:

\begin{itemize}
    \item corrected bounded viability normalization,
    \item smooth decay functional with structural coupling,
    \item proper distinction between finite-time and asymptotic collapse,
    \item rigorously defined collapse manifold,
    \item terminal absorbing behavior after collapse,
    \item full compatibility with curvature, metric, and temporal models.
\end{itemize}

Viability is the field that determines life, degeneration, and death of the market organism.

% ============================================================
% 6. Dynamic Regimes
% ============================================================
% ============================================================
% 6. Dynamic Regimes: ACC, DEV, REL, COL
% ============================================================
\section{Dynamic Regimes: ACC, DEV, REL, COL}

The structural evolution of the Flexion Market Organism proceeds through four qualitatively distinct regimes:
\[
ACC \;\rightarrow\; DEV \;\rightarrow\; REL \;\rightarrow\; COL.
\]
FMT~3.1 promotes this ordering from a derived observation to a fundamental axiom.
Regime transitions arise solely from internal geometric interactions among
\(\Delta, \Phi, M, \kappa\), curvature \(R(X)\), and temporal density \(\tau(X)\).
No external market inputs participate.

Regimes describe the large-scale physiological phases of the organism’s life cycle.

% ------------------------------------------------------------
\subsection{6.1 Regime Ordering Axiom}

There exists a monotonic structural regime index
\[
\mathcal{R}(X) \in \{0,1,2,3\},
\]
such that:
\[
\mathcal{R}(X)=0 \Rightarrow ACC,\quad
\mathcal{R}(X)=1 \Rightarrow DEV,\quad
\mathcal{R}(X)=2 \Rightarrow REL,\quad
\mathcal{R}(X)=3 \Rightarrow COL,
\]
and for all living states,
\[
\mathcal{R}(X(t+1)) \ge \mathcal{R}(X(t)).
\]

Regime reversal is forbidden.  
This irreversibility defines the organism’s developmental arrow of time.

% ------------------------------------------------------------
\subsection{6.2 Regime Determination Principles}

Regimes are determined by:

\begin{itemize}
    \item sign patterns of structural derivatives  
    \( d\|\Delta\|/dt,\, d\Phi/dt,\, dM/dt,\, d\kappa/dt \);
    \item curvature intensity and growth rate;
    \item viability load and decay functional \(\Pi(X)\);
    \item temporal density behavior.
\end{itemize}

These principles yield smooth, well-defined transition surfaces between regimes.

% ------------------------------------------------------------
\subsection{6.3 ACC — Accumulation Regime}

ACC is the organism’s initial stress–accumulation phase.

Structural conditions:
\[
\frac{d\Phi}{dt} > 0,
\qquad
\frac{d\|\Delta\|}{dt} \ge 0,
\qquad
R \text{ small or moderately increasing},
\qquad
\kappa \approx \kappa_{\max}.
\]

Interpretation:
\begin{itemize}
    \item tension accumulates,
    \item geometric asymmetry begins forming,
    \item memory grows steadily,
    \item viability remains high.
\end{itemize}

Transition criterion (ACC \(\rightarrow\) DEV):
\[
\frac{d\|\Delta\|}{dt} > \theta_{\Delta}\,\frac{d\Phi}{dt},
\]
with structural threshold \(\theta_{\Delta}>0\).

% ------------------------------------------------------------
\subsection{6.4 DEV — Development Regime}

DEV represents geometric expansion and structural differentiation.

Structural conditions:
\[
\frac{d\|\Delta\|}{dt} > 0, \qquad
R \text{ increasing}, \qquad
\Phi \text{ stabilizing}, \qquad
\kappa \text{ decreasing sublinearly}.
\]

Interpretation:
\begin{itemize}
    \item deformation grows rapidly,
    \item curvature amplifies structural instability,
    \item memory accumulation accelerates long-term drift.
\end{itemize}

Transition criterion (DEV \(\rightarrow\) REL):
\[
\frac{d\Phi}{dt} < 0.
\]

% ------------------------------------------------------------
\subsection{6.5 REL — Relaxation Regime}

REL is the structural reorganization and tension-dissipation phase.

Structural conditions:
\[
\frac{d\Phi}{dt} < 0,
\qquad
\frac{d\|\Delta\|}{dt} \approx 0,
\qquad
R \text{ moderate or slowly increasing},
\qquad
\tau \text{ decreasing}.
\]

Interpretation:
\begin{itemize}
    \item accumulated tension dissipates,
    \item the organism stabilizes deformation,
    \item viability decays more noticeably,
    \item temporal flow slows.
\end{itemize}

Transition criterion (REL \(\rightarrow\) COL):
\[
\Pi(X) > \Pi_{\mathrm{crit}},
\]
where \(\Pi_{\mathrm{crit}}\) is the structural collapse threshold determined by curvature and viability.

% ------------------------------------------------------------
\subsection{6.6 COL — Collapse Regime}

COL is the terminal degenerative phase where the organism approaches the collapse manifold.

Structural conditions:
\[
\kappa \to 0, \qquad
R \to \infty, \qquad
\det g \to 0, \qquad
\tau \to 0.
\]

Interpretation:
\begin{itemize}
    \item geometric capacity disappears,
    \item curvature diverges,
    \item structural time stagnates,
    \item morphology saturates (\(\mu \to 1\)).
\end{itemize}

Collapse occurs at the first moment \(t_c\) when \(\kappa(t_c)=0\).

% ------------------------------------------------------------
\subsection{6.7 Regime Transition Surfaces}

The regime boundaries form smooth surfaces in structural space:

\begin{itemize}
    \item ACC \(\rightarrow\) DEV:  
    \[
    \frac{d\|\Delta\|}{dt} = 
    \theta_{\Delta}\,\frac{d\Phi}{dt}.
    \]

    \item DEV \(\rightarrow\) REL:  
    \[
    \frac{d\Phi}{dt} = 0.
    \]

    \item REL \(\rightarrow\) COL:  
    \[
    \Pi(X) = \Pi_{\mathrm{crit}}.
    \]
\end{itemize}

These transition surfaces are continuous, non-self-intersecting, and respect the regime axiom.

% ------------------------------------------------------------
\subsection{6.8 Regimes and Morphology}

Morphology and regimes are correlated but not identical:

\begin{itemize}
    \item Elastic \(\leftrightarrow\) early ACC,
    \item Plastic \(\leftrightarrow\) late ACC / DEV,
    \item Degenerate \(\leftrightarrow\) DEV / REL,
    \item NearCollapse \(\leftrightarrow\) REL / COL.
\end{itemize}

FMT~3.1 removes the incorrect monotonicity constraint on morphology, allowing internal fluctuations within regimes.

% ------------------------------------------------------------
\subsection{6.9 Summary}

FMT~3.1 establishes a complete, consistent, and irreversible regime structure:

\begin{itemize}
    \item regime ordering is axiomatic,
    \item transitions are defined by geometric criteria,
    \item morphology aligns with regimes without strict equivalence,
    \item collapse is triggered by viability and curvature dynamics,
    \item all inconsistencies from earlier versions are eliminated.
\end{itemize}

These regimes describe the full physiological life cycle of the market organism.

% ============================================================
% 7. Morphology and Morphology Index
% ============================================================
% ============================================================
% 7. Morphology and Morphology Index μ(X)
% ============================================================
\section{Morphology and Morphology Index \texorpdfstring{$\mu(X)$}{μ(X)}}

Morphology provides a compressed geometric–viability descriptor of the organism’s structural state.
It summarizes contributions from curvature, viability, tension, and memory into a single scalar in
the interval
\[
\mu(X) \in [0,1].
\]

FMT~3.1 introduces corrected normalization functions, resolves inconsistencies from earlier versions,
and ensures smooth, collapse-consistent behavior.

% ------------------------------------------------------------
\subsection{7.1 Morphology Index Definition}

The morphology index is defined as
\[
\mu(X)
=
\omega_R \sigma_R(R)
+ \omega_\kappa \sigma_\kappa(\kappa)
+ \omega_\Phi \sigma_\Phi(\Phi)
+ \omega_M \sigma_M(M),
\]
with weights
\[
\omega_i \ge 0,
\qquad
\sum_{i\in\{R,\kappa,\Phi,M\}} \omega_i = 1.
\]

Each \(\sigma_i\) is a smooth normalization function mapping its argument to \([0,1]\).
Morphology increases with:

\begin{itemize}
    \item curvature intensity,
    \item structural stress,
    \item accumulated memory,
    \item viability depletion.
\end{itemize}

% ------------------------------------------------------------
\subsection{7.2 Normalization Functions}

FMT~3.1 adopts corrected bounded, smooth, and collapse-compatible normalization functions:

\paragraph{Curvature normalization}
\[
\sigma_R(R) = \frac{R}{1 + R}.
\]

\paragraph{Tension normalization}
\[
\sigma_\Phi(\Phi) = \frac{\Phi}{1 + \Phi}.
\]

\paragraph{Memory normalization}
\[
\sigma_M(M) = \frac{M}{1 + M}.
\]

\paragraph{Corrected viability normalization}
The prior linear form was inconsistent with unbounded viability.
FMT~3.1 uses:
\[
\sigma_\kappa(\kappa) = 1 - e^{-\lambda \kappa}, \qquad \lambda > 0.
\]
This ensures:

\begin{itemize}
    \item proper boundedness (\(\sigma_\kappa \in [0,1)\)),
    \item smoothness across the living domain,
    \item compatibility with collapse limits,
    \item global consistency of \(\mu(X)\in[0,1]\).
\end{itemize}

This correction resolves the single largest mathematical flaw in FMT~3.0.

% ------------------------------------------------------------
\subsection{7.3 Geometric Interpretation}

Morphology reflects the structural “health” of the organism:

\begin{itemize}
    \item low $\mu$: elastic, stable geometry,
    \item moderate $\mu$: plastic, progressively deforming structure,
    \item high $\mu$: degenerate, unstable configuration,
    \item $\mu \to 1$: near collapse.
\end{itemize}

Importantly, FMT~3.1 removes the incorrect monotonicity assumption from earlier versions:
\[
\mu(t+1) \ge \mu(t) \quad \text{is not required.}
\]
Morphology may fluctuate within regimes (especially REL), while still trending upward toward collapse.

% ------------------------------------------------------------
\subsection{7.4 Morphological Classes}

Morphology is categorized into structural classes:

\[
\text{Elastic},\quad
\text{Plastic},\quad
\text{Degenerate},\quad
\text{NearCollapse}.
\]

Representative thresholds:
\[
\mu < \mu_1 \Rightarrow \text{Elastic},\qquad
\mu_1 \le \mu < \mu_2 \Rightarrow \text{Plastic},
\]
\[
\mu_2 \le \mu < \mu_3 \Rightarrow \text{Degenerate},\qquad
\mu \ge \mu_3 \Rightarrow \text{NearCollapse}.
\]

These classes correspond to qualitative geometric behavior:

\begin{itemize}
    \item \textbf{Elastic}: low curvature, high viability,
    \item \textbf{Plastic}: moderate stress and deformation,
    \item \textbf{Degenerate}: elevated curvature, decreasing viability,
    \item \textbf{NearCollapse}: curvature diverging, viability approaching zero.
\end{itemize}

% ------------------------------------------------------------
\subsection{7.5 Morphology–Regime Relationship}

Morphology correlates with—but does not strictly define—regimes:

\begin{itemize}
    \item Elastic → mostly ACC,
    \item Plastic → late ACC / DEV,
    \item Degenerate → DEV / REL,
    \item NearCollapse → REL / COL.
\end{itemize}

FMT~3.1 explicitly prohibits forcing regime boundaries to align with morphology thresholds.

Morphology is a geometric descriptor, not a regime classifier.

% ------------------------------------------------------------
\subsection{7.6 Smoothness and Collapse Behavior}

FMT~3.1 ensures:

\begin{itemize}
    \item continuity of $\mu(X)$ for all $\kappa>0$,
    \item differentiability from smooth $\sigma_i$,
    \item correct collapse asymptotics.
\end{itemize}

At collapse:
\[
\sigma_R(R)\to 1, \qquad
\sigma_\kappa(\kappa)\to 1, \qquad
\sigma_M(M)\to 1, \qquad
\sigma_\Phi(\Phi)\to 1,
\]
implying the universal limit:
\[
\boxed{\lim_{\kappa\to 0} \mu(X) = 1.}
\]

Thus morphology provides a unified, collapse-consistent measure of structural degradation.

% ------------------------------------------------------------
\subsection{7.7 Summary}

FMT~3.1 introduces a fully corrected morphology framework:

\begin{itemize}
    \item smooth, bounded normalization functions,
    \item corrected viability normalization,
    \item removal of false monotonicity,
    \item consistent morphological classes,
    \item collapse limit $\mu\to 1$ ensured by geometry,
    \item complete compatibility with curvature, viability, and metric models.
\end{itemize}

Morphology serves as a compact structural fingerprint of the organism at all stages of its evolution.

% ============================================================
% 8. Structural Invariants
% ============================================================
% ============================================================
% 8. Structural Invariants
% ============================================================
\section{Structural Invariants}

Structural invariants impose fundamental constraints on the evolution of the Flexion Market Organism.
They ensure internal consistency, preserve geometric validity, and guarantee that collapse dynamics
follow the correct mathematical limits.  
FMT~3.1 clarifies, corrects, and formally strengthens the invariant system.

All invariants apply strictly within the living domain:
\[
\kappa>0,\qquad \det g(X)>0,\qquad \tau(X)>0.
\]

% ------------------------------------------------------------
\subsection{Invariant I: Memory Irreversibility}

Memory defines structural time, therefore:
\[
M(t+1) \ge M(t).
\]

Equivalent continuous form:
\[
\frac{dM}{dt} = \tau(X) > 0 \quad \text{for all living states}.
\]

This invariant establishes the intrinsic arrow of time.

% ------------------------------------------------------------
\subsection{Invariant II: Viability Non-Negativity}

Viability cannot become negative:
\[
\kappa(t+1) \ge 0.
\]

Collapse occurs at the boundary:
\[
\kappa(t_c)=0.
\]

No evolution is defined beyond this boundary.

% ------------------------------------------------------------
\subsection{Invariant III: Metric Positivity}

For all living states:
\[
\det g(X) > 0.
\]

The metric degenerates smoothly at collapse:
\[
\det g(X) \to 0 
\quad \text{as}\quad 
\kappa \to 0.
\]

This ensures geometric consistency across the entire lifespan.

% ------------------------------------------------------------
\subsection{Invariant IV: Temporal Density Positivity}

Temporal density satisfies:
\[
\tau(X) > 0,
\qquad
\tau \to 0 \quad \text{as }\kappa\to 0.
\]

This guarantees smooth slowdown of structural time toward collapse.

% ------------------------------------------------------------
\subsection{Invariant V: Regime Irreversibility}

The regime index \(\mathcal{R}(X) \in \{0,1,2,3\}\) satisfies:
\[
\mathcal{R}(t+1) \ge \mathcal{R}(t).
\]

Regimes follow the irreversible ordering:
\[
ACC \rightarrow DEV \rightarrow REL \rightarrow COL.
\]

Reversal is forbidden.

% ------------------------------------------------------------
\subsection{Invariant VI: Continuity of Evolution}

The evolution operator \(I(X)\) must satisfy:
\[
\lim_{\varepsilon\to 0}
\left\| I(X+\varepsilon) - I(X) \right\| = 0,
\]
for all \(\kappa>0\).

This continuity is necessary to ensure smooth dynamics of curvature, metric, and morphology.

% ------------------------------------------------------------
\subsection{Invariant VII: Finiteness of Structural Quantities}

All structural quantities remain finite for \(\kappa>0\):
\[
\|\Delta\| < \infty,\quad 
\Phi < \infty,\quad 
M < \infty,\quad 
R < \infty.
\]

Curvature diverges only at collapse.

This removes hidden divergences present in earlier versions.

% ------------------------------------------------------------
\subsection{Invariant VIII: Collapse Geometry}

As the organism approaches collapse:
\[
R(X) \to \infty,\qquad
\det g(X) \to 0,\qquad
\tau(X) \to 0,\qquad
\mu(X) \to 1.
\]

These limits must be respected by all structural equations.

FMT~3.1 ensures that collapse geometry is smooth, non-singular for \(\kappa>0\),  
and fully compatible with the viability decay functional.

% ------------------------------------------------------------
\subsection{Invariant IX: Autonomy}

Structural evolution depends only on the internal state:
\[
\frac{\partial I}{\partial \text{external}} = 0.
\]

No external market signals influence the organism’s internal dynamics.

% ------------------------------------------------------------
\subsection{8.10 Summary}

FMT~3.1 establishes a complete system of structural invariants that ensure:

\begin{itemize}
    \item irreversible temporal flow,
    \item non-negative viability,
    \item positive structural metric for all living states,
    \item continuous, finite geometric evolution,
    \item correct asymptotic limits at collapse,
    \item strict regime irreversibility,
    \item full autonomy of internal dynamics.
\end{itemize}

These invariants provide the essential constraints for all subsequent equations of motion.

% ============================================================
% 9. Structural Equations of Motion
% ============================================================
% ============================================================
% 9. Structural Equations of Motion
% ============================================================
\section{Structural Equations of Motion}

The structural equations of motion describe how the organism’s internal fields evolve under geometric
load, memory accumulation, and viability decay.  
FMT~3.1 provides corrected, consistent, and collapse-compatible formulations that follow directly 
from the invariants and geometric definitions.

Let the structural state be
\[
X(t) = (\Delta(t),\, \Phi(t),\, M(t),\, \kappa(t)),
\]
and let \(I(X)\) denote the evolution operator:
\[
X(t+1) = I(X(t)).
\]

% ------------------------------------------------------------
\subsection{9.1 Deformation–Differentiation Law}

Deformation evolves according to internal differentiation pressure:
\[
\Delta(t+1)
=
\Delta(t) + \mathcal{D}(X(t)).
\]

The deformation functional \(\mathcal{D}\) satisfies:

\begin{itemize}
    \item continuity on the living domain,
    \item increasing response to curvature and tension,
    \item boundedness: \(\|\mathcal{D}(X)\| < \infty\) for \(\kappa > 0\),
    \item collapse limit: \(\|\mathcal{D}(X)\| \to \infty\) as \(\kappa \to 0\).
\end{itemize}

A canonical form is:
\[
\mathcal{D}(X) = 
d_\Delta \Delta + 
d_\Phi \Phi\,u_\Delta +
d_R\,\sigma_R(R)\,u_R,
\]
where \(u_\Delta, u_R\) are normalized direction vectors.

% ------------------------------------------------------------
\subsection{9.2 Tension Law}

Tension responds to geometric deformation and accumulated curvature:
\[
\Phi(t+1)
=
\Phi(t) + \mathcal{T}(X(t)).
\]

A consistent canonical form is:
\[
\mathcal{T}(X)
=
t_\Delta \|\Delta\|
+ t_R \sigma_R(R)
- t_{\mathrm{rel}}\,\sigma_{\mathrm{rel}}(\text{REL-phase}),
\]
where the last term allows tension dissipation in the REL regime.

Tension remains finite for all living states:
\[
\Phi < \infty \quad \text{for } \kappa>0.
\]

% ------------------------------------------------------------
\subsection{9.3 Structural Time Law}

Memory evolves according to temporal density:
\[
M(t+1) = M(t) + \tau(X(t)),
\qquad
\tau(X) > 0 \text{ for all living states}.
\]

A canonical temporal-density model:
\[
\tau(X) =
\tau_0 e^{-\gamma \kappa}
+ \tau_\Phi\,\sigma_\Phi(\Phi)
+ \tau_R\,\sigma_R(R),
\]
satisfying
\[
\tau(X) \to 0 \quad \text{as } \kappa\to 0.
\]

This generates irreversible structural time.

% ------------------------------------------------------------
\subsection{9.4 Viability Law}

Viability decays as:
\[
\kappa(t+1)
=
\kappa(t) - \Pi(X(t)),
\]
where \(\Pi(X)\) is the decay functional.

A canonical fully consistent form:
\[
\Pi(X)
=
a_R \sigma_R(R)
+ a_\Phi \sigma_\Phi(\Phi)
+ a_M \sigma_M(M)
+ a_\Delta \sigma_\Delta(\|\Delta\|)
+ a_\kappa \kappa^{-\beta},
\qquad \beta>0.
\]

This structure supports both:

- **finite-time collapse** (if \(\Pi \ge \varepsilon > 0\)),
- **asymptotic collapse** (if \(\Pi(X)\to 0\) as \(\kappa\to 0\)).

% ------------------------------------------------------------
\subsection{9.5 Curvature and Metric Equations}

Curvature:
\[
R(X) =
A\|\Delta\|^2
+ B\Phi
+ C M
+ D\,\kappa^{-\alpha},
\qquad
\alpha > 0.
\]

Metric:
\[
\det g(X)
=
g_0 - c_R R(X),
\qquad
c_R > 0.
\]

Required properties:

\[
R < \infty,\quad \det g > 0 \quad \text{for all } \kappa>0,
\]
\[
R\to\infty,\quad \det g\to 0 \quad \text{as } \kappa\to 0.
\]

These ensure proper geometric degeneration near collapse.

% ------------------------------------------------------------
\subsection{9.6 Morphology Equation}

Morphology is defined as:
\[
\mu(X)
=
\omega_R \sigma_R(R)
+ \omega_\kappa \sigma_\kappa(\kappa)
+ \omega_\Phi \sigma_\Phi(\Phi)
+ \omega_M \sigma_M(M).
\]

Required limits:
\[
\mu(X)\in[0,1] \quad (\kappa>0),
\qquad
\lim_{\kappa\to 0} \mu(X) = 1.
\]

Morphology is not required to be monotonic in time.

% ------------------------------------------------------------
\subsection{9.7 Regime Dynamics}

Regime index:
\[
\mathcal{R}(X) \in \{0,1,2,3\},
\]
with the irreversible progression:
\[
ACC \rightarrow DEV \rightarrow REL \rightarrow COL.
\]

Transition criteria:

\begin{itemize}
    \item ACC→DEV:  
    \[
    \frac{d\|\Delta\|}{dt}
    =
    \theta_\Delta \frac{d\Phi}{dt}.
    \]

    \item DEV→REL:  
    \[
    \frac{d\Phi}{dt} = 0.
    \]

    \item REL→COL:  
    \[
    \Pi(X) = \Pi_{\mathrm{crit}}.
    \]
\end{itemize}

% ------------------------------------------------------------
\subsection{9.8 Unified System Summary}

The complete discrete-time evolution system is:
\[
\boxed{
\begin{aligned}
\Delta(t+1) &= \Delta(t) + \mathcal{D}(X(t)), \\
\Phi(t+1)   &= \Phi(t) + \mathcal{T}(X(t)), \\
M(t+1)     &= M(t) + \tau(X(t)), \\
\kappa(t+1)&= \kappa(t) - \Pi(X(t)), \\
\end{aligned}}
\]

with geometric definitions:
\[
R(X) = 
A\|\Delta\|^2 + B\Phi + CM + D\,\kappa^{-\alpha},
\]
\[
\det g(X) = g_0 - c_R R(X),
\]
\[
\mu(X)
= 
\omega_R\sigma_R(R)
+ \omega_\kappa\sigma_\kappa(\kappa)
+ \omega_\Phi\sigma_\Phi(\Phi)
+ \omega_M\sigma_M(M).
\]

Collapse occurs at:
\[
\kappa(t_c)=0.
\]

This system forms the complete structural physics of FMT~3.1.

% ============================================================
% 10. Collapse Geometry and Terminal Behavior
% ============================================================
% ============================================================
% 10. Collapse Geometry and Terminal Behavior
% ============================================================
\section{Collapse Geometry and Terminal Behavior}

Collapse represents the terminal state of the Flexion Market Organism.  
It occurs when viability reaches zero:
\[
\kappa(t_c) = 0.
\]

Near the collapse boundary, all geometric and structural fields obey precise asymptotic limits.  
FMT~3.1 provides a corrected, consistent formulation of collapse geometry that resolves
all inconsistencies from earlier versions.

% ------------------------------------------------------------
\subsection{10.1 Collapse Boundary}

The collapse manifold is defined as:
\[
\mathcal{C} = \{X : \kappa = 0\}.
\]

The living domain is:
\[
\mathcal{D}_{\mathrm{alive}} = \{X : \kappa > 0,\ \det g(X) > 0,\ \tau(X) > 0\}.
\]

Collapse occurs when the boundary \(\mathcal{C}\) is first reached.

% ------------------------------------------------------------
\subsection{10.2 Collapse Dynamics: Geometric Limits}

As \(\kappa \to 0\), the organism’s geometry degenerates according to the following universal limits:

\paragraph{Curvature Divergence}
\[
R(X) \to \infty.
\]

\paragraph{Metric Degeneration}
\[
\det g(X) \to 0.
\]

\paragraph{Temporal Density Collapse}
\[
\tau(X) \to 0.
\]

\paragraph{Morphology Saturation}
\[
\mu(X) \to 1.
\]

These limits collectively characterize the collapse geometry, forming a smooth and internally consistent degenerative structure.

% ------------------------------------------------------------
\subsection{10.3 Curvature Divergence}

Curvature diverges as:
\[
R(X) = 
A\|\Delta\|^2 + B\Phi + C M + D\,\kappa^{-\alpha},
\quad \alpha>0,
\]
which yields:
\[
\lim_{\kappa \to 0} R(X) = +\infty.
\]

Curvature divergence is the primary mechanism of structural breakdown.

% ------------------------------------------------------------
\subsection{10.4 Metric Degeneration}

The structural metric
\[
\det g(X) = g_0 - c_R R(X)
\]
satisfies:
\[
\det g(X) > 0 \;\text{for} \;\kappa>0,
\qquad
\det g(X) \to 0 \;\text{as}\;\kappa\to 0.
\]

Metric degeneration reflects the collapse of structural geometric capacity.

% ------------------------------------------------------------
\subsection{10.5 Temporal Collapse}

Temporal density approaches zero:
\[
\tau(X) = 
\tau_0 e^{-\gamma \kappa}
+ \tau_{\Phi} \sigma_{\Phi}(\Phi)
+ \tau_R \sigma_R(R)
\to 0.
\]

Consequences:
\[
M(t+1) - M(t) \to 0,
\]
\[
\frac{dT_s}{dt} = \tau(X) \to 0.
\]

Structural time stagnates as collapse is approached.

% ------------------------------------------------------------
\subsection{10.6 Morphological Terminal State}

As collapse nears:
\[
\sigma_R(R)\to 1,\quad
\sigma_\kappa(\kappa)\to 1,\quad
\sigma_\Phi(\Phi)\to 1,\quad
\sigma_M(M)\to 1.
\]

Thus:
\[
\mu(X) = 
\omega_R \sigma_R(R)
+ \omega_\kappa \sigma_\kappa(\kappa)
+ \omega_\Phi \sigma_\Phi(\Phi)
+ \omega_M \sigma_M(M)
\;\;\longrightarrow\;\; 1.
\]

Morphology reaches full degeneration.

% ------------------------------------------------------------
\subsection{10.7 Collapse Modalities}

FMT~3.1 distinguishes two fundamentally different collapse modalities:

\paragraph{Finite-time Collapse}
\[
\exists\ \varepsilon>0 :\ \Pi(X)\ge \varepsilon \quad \Rightarrow\quad \kappa(t_c)=0.
\]

Viability reaches zero in finite structural time.

\paragraph{Asymptotic Collapse}
\[
\Pi(X)\to 0 \quad \text{as}\quad \kappa\to 0.
\]

Then:
\[
\lim_{t\to\infty} \kappa(t)=0,
\]
but collapse is never reached in finite structural time.

This correction is one of the central improvements of FMT~3.1.

% ------------------------------------------------------------
\subsection{10.8 Terminal Absorbing State}

After collapse (\(t \ge t_c\)):

\[
X(t) = X(t_c),
\]
\[
\Delta, \Phi, M, \kappa \text{ become constant},
\]
\[
R = \infty, \qquad 
\det g = 0, \qquad 
\tau = 0, \qquad 
\mu = 1.
\]

No further structural evolution is possible.  
Collapse is the mathematically final and absorbing endpoint of the organism’s life cycle.

% ------------------------------------------------------------
\subsection{10.9 Summary}

FMT~3.1 provides a complete collapse geometry:
\begin{itemize}
    \item precise collapse boundary \(\kappa=0\),
    \item curvature divergence and metric degeneration,
    \item temporal stagnation as collapse is approached,
    \item morphology saturation to unity,
    \item corrected treatment of finite vs asymptotic collapse,
    \item a terminal absorbing state consistent with all invariants.
\end{itemize}

Collapse is not a failure mode but an intrinsic geometrical completion of structural evolution.

% ============================================================
% 11. Unified Structural Interpretation
% ============================================================
% ============================================================
% 11. Unified Structural Interpretation
% ============================================================
\section{Unified Structural Interpretation}

Flexion Market Theory describes the market as a self-contained geometric organism whose evolution,
instabilities, and eventual collapse arise entirely from internal structural interactions.
This section integrates the previously defined components into a unified conceptual model.

% ------------------------------------------------------------
\subsection{11.1 Geometry of Internal Forces}

All structural forces—deformation, tension, memory, and viability—manifest as geometric phenomena.
The organism evolves within a self-generated manifold with curvature
\[
R(X) = A\|\Delta\|^2 + B\Phi + C M + D\,\kappa^{-\alpha}.
\]

Curvature determines the organism’s internal “pressure”:

\begin{itemize}
    \item $\|\Delta\|$ introduces directional deformation,
    \item $\Phi$ encodes accumulated stress,
    \item $M$ amplifies long-term drift,
    \item $\kappa$ regulates geometric instability.
\end{itemize}

Geometric forces, not external market signals, drive evolution.

% ------------------------------------------------------------
\subsection{11.2 Temporal Interpretation}

Structural time is an intrinsic quantity generated by memory:
\[
T_s(t) = M(t), \qquad \frac{dT_s}{dt} = \tau(X).
\]

Interpretation:

\begin{itemize}
    \item FAST internal time $\leftrightarrow$ high stress and curvature,
    \item SLOW internal time $\leftrightarrow$ diminishing viability,
    \item STOPPED time $\leftrightarrow$ collapse ($\tau=0$).
\end{itemize}

The organism ages by accumulating memory; no external clock is relevant.

% ------------------------------------------------------------
\subsection{11.3 Life Cycle of the Organism}

The four regimes form a complete physiological cycle:

\begin{enumerate}
    \item \textbf{ACC} — stress accumulation and early asymmetry,
    \item \textbf{DEV} — directional growth and geometric intensification,
    \item \textbf{REL} — dissipation and partial stabilization,
    \item \textbf{COL} — terminal degeneration and collapse.
\end{enumerate}

This trajectory is irreversible:
\[
ACC \rightarrow DEV \rightarrow REL \rightarrow COL.
\]

Each regime expresses a distinct geometric function of the organism.

% ------------------------------------------------------------
\subsection{11.4 Morphology as Structural Health}

Morphology serves as a synthetic measure of structural degradation:
\[
\mu(X)
=
\omega_R \sigma_R(R)
+ \omega_\kappa \sigma_\kappa(\kappa)
+ \omega_\Phi \sigma_\Phi(\Phi)
+ \omega_M \sigma_M(M).
\]

Interpretation:

\begin{itemize}
    \item $\mu \approx 0$ — elastic, healthy structure,
    \item $\mu \approx 0.5$ — plastic or transitional,
    \item $\mu \approx 0.8$ — degenerating structure,
    \item $\mu \to 1$ — near collapse.
\end{itemize}

Morphology reflects the global state of the geometry, not the local regime.

% ------------------------------------------------------------
\subsection{11.5 Collapse as Structural Death}

Collapse is not an external event but the natural geometric termination of evolution:
\[
\kappa = 0,\quad R\to\infty,\quad \det g\to 0,\quad \tau\to 0,\quad \mu\to 1.
\]

At collapse:

\begin{itemize}
    \item the manifold ceases to exist,
    \item structural time stops,
    \item memory accumulation halts,
    \item all state components freeze:
    \[
    X(t) = X(t_c),\quad t\ge t_c.
    \]
\end{itemize}

Collapse is the organism’s mathematically final absorbing state.

% ------------------------------------------------------------
\subsection{11.6 Relationship to Other Flexion Theories}

FMT~3.1 fits coherently into the broader Flexion Universe:

\begin{itemize}
    \item \textbf{Flexion Framework V1.5}  
    — defines the four structural fields and global invariants.

    \item \textbf{Flexion Space Theory}  
    — governs the geometric manifold and curvature behavior.

    \item \textbf{Flexion Time Theory}  
    — formalizes temporal density and structural time.

    \item \textbf{Flexion Field Theory}  
    — provides field-level interpretations of $\Delta$, $\Phi$, and $R$.

    \item \textbf{Flexion Entanglement Theory}  
    — describes interactions between multiple organisms.

    \item \textbf{FMRT (Runtime Engine)}  
    — computationally implements the evolution operator \(I(X)\).
\end{itemize}

FMT provides the dynamics of a \textit{single} market organism;  
multi-organism interactions are addressed in FET.

% ------------------------------------------------------------
\subsection{11.7 Unified Interpretation Summary}

Flexion Market Theory presents the market as a geometric lifeform whose evolution is governed by:

\begin{itemize}
    \item internal forces encoded in $\Delta$, $\Phi$, $M$, and $\kappa$,
    \item geometric constraints expressed through curvature and metric,
    \item intrinsic structural time driven by memory,
    \item irreversible regime progression,
    \item morphological degradation summarizing structural health,
    \item terminal collapse defined by precise geometric limits.
\end{itemize}

FMT~3.1 provides a complete and unified physical description of the market organism.

% ============================================================
% References
% ============================================================
\begin{thebibliography}{9}

\bibitem{FMT31}
Maryan Bogdanov, \textit{Flexion Market Theory V3.1}, 2025.

\end{thebibliography}

\end{document}
